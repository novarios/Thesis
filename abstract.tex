\documentclass[../thesis.tex]{subfiles}

\begin{document}

Coupled Cluster theory is a powerful \emph{ab initio} framework for solving the many-body the Schr\"{o}dinger equation and has been utilized successfully to describe the highly-correlated systems found in quantum chemistry and nuclear physics.  This method uses a special similarity transformation to decouple a system's ground state from excitations from it.  This transformation contains significant correlations that can be used to extend coupled cluster theory to excited states and open-shell systems with the equation-of-motions method.  Additionally, properites of these states can be obtained by consistently transforming relevant operators using the coupled cluster similarity transfomation.  The coupled cluster method is systematically improvable and scales polynomially with the system size.  With this flexibility and reach, coupled cluster theory can be applied across the nuclear chart to contribute to many important open problems in physics.

Several fundamental questions in modern physics involve electroweak interactions within nuclei, including the search for the elusive neutrinoless double-beta decay.  Often the largest uncertainy within these experiments is due to nuclear-structure-dependent quantities that are calculated within some many-body framework.  The main focus of this thesis is to apply the coupled cluster method to calculate effective Fermi and Gamow-Teller beta-decay operators between open shell states.  By confirming the validity of this method, it can be extended to double-beta decay and other electroweak processes.

\end{document}
