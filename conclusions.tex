\documentclass[../thesis.tex]{subfiles}

\begin{document}

\section{Summary and Conclusions}

Accurate \emph{ab initio} calculations of beta-decay transition amplitudes are necessary for answering many open questions from a wide range of areas in modern physics from nucleosynthesis to fundamental symmetries.  The accuracy and scope needed to answer such questions necessitate a technique that is widely applicable, systematically improvable, and scalable to large systems.  In this thesis, we have developed the formalism for and achieved the application of techniques based on coupled cluster theory that fulfill these requirements.

The large and versatile program that implements these techniques can be used for many different fermionic systems including the homogeneous electron gas, quantum dots, and finite nuclei.  Because of the program's modular form, additional systems like neutron drops, infinite nuclear matter, and atomic systems can easily be added in subsequent updates.  Importantly, we extended the single-determinant coupled-cluster method to open-shell systems using the equation-of-motion method which grants a broader reach across the nuclear chart.  Also, we added the crucial ability to perform calculations with and without three-body forces which ensures accurate results.  Also modular, these components of the program can be easily extended to higher-order EOM approximations, two-particle-attached and two-particle-removed EOM states, and the inclusion of full three-body forces.  Lastly, we've implemented the ability to construct any effective one-body operator and calculate the corresponding observables using ground and excited EOM states.  In future iterations of this code, higher-order effective operators, like those required for double-beta-decay experiments, can also be implemented.  Also, these higher-order operators can be constructed from two-body chiral weak currents and used to investigate the quenching of the axial vector coupling constant.

In addition to developing and implementing these various techniques, we performed calculations at each step to verify the results.  In particular, we provided a proof of principle by comparing our results with those from other \emph{ab initio} methods for various different systems.  Also, we calculated ground-state energies as well as particle-attached and particle-removed spectra for various light nuclei, focusing mainly on the oxygen chain.  In future publications, this machinery will be extended up the nuclear chart to heavier nuclei and out to the limits of stability to calculate beta-decay properties of nuclei around $^{78}$Ni and $^{100}$Sn, which are important to future experiments at FRIB.  For example, consistent beta-decay lifetime calculations from \emph{ab initio} methods will be invaluable for astrophysical simulations of different nucleosynthesis processes.




\end{document}
