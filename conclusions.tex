\documentclass[../thesis.tex]{subfiles}

\begin{document}

\section{Summary and Conclusions}

This work focused on the coupled cluster method and its extension to open-shell nuclei.  First, the CC algorithm with performed for various systems, including several closed-shell nuclei.  After a full treatment of the nuclear problem, with the inclusion of three-body forces and center-of-mass factorization, these CC solutions were used to extend the reach of coupled cluster to open-shell nuclei.  In particular, several particle-attached and particle-removed nuclei were studied, solving for the the ground and excited states.  Then the center-of-mass diagnostic was used to remove spurious states from the spectrum.  Lastly, the CC solutions were used to construct effective Gamow-Teller and Fermi beta-decay operators.  Most calculations centered on closed-shell oxygen isotopes and their particle-attached and particle-removed neighbors.

While this work demonstrated many qualitative features of the various quantities, it suffered from several shortcomings that can be remedied in the future.  First, a lack of sufficient computing resources restricted the model space to a relatively small size such that the results never fully converged.  Additionally, time constraints hindered a full exploration of all the available interactions and model-space parameters.  Accounting for these drawbracks, the techniques presented in this thesis show great promise as an effective and flexible tool with which to explore the beta-decay properties across the nuclear chart.  

\end{document}
