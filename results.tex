\documentclass[thesis.tex]{subfiles}

\section{Center-of-Mass Factorization}
For atomic systems, the single-particle orbitals are built from the essentially fixed Coulomb potential of the nucleus.  Conversely, self-bound nuclear systems are translationally invariant, but the single-particle orbits from which they are built are not.  Therefore, there is an added complication with nuclear systems where the intrinsic state and the center-of-mass state are not neccessarily seperate.  A fully seperated ground-state wavefunction can be written as,
\begin{equation}
  \corrket = \corrket_{\text{cm}}\corrket_{\text{in}},
\end{equation}
where $\corrket_{\text{cm}}$ is the center-of-mass wavefunction and $\corrket_{\text{in}}$ is the intrinsic wavefunction.

\begin{equation}
  \Ham_{\text{in}} = \Top - \Top_{\text{cm}} + \Vop = \sum_{i<j}^{A} \left[ \frac{\left( \hat{\mathbf{p}}_{i} - \hat{\mathbf{p}}_{j} \right)^{2}}{2mA} + \Vop\left( \mathbf{r}_{i} - \mathbf{r}_{j} \right) \right]
\end{equation}
By defining the center-of-mass momentum as $\vec{P}=\sum_{i=1}^{A}\vec{p}_{i}$, this separation of the kinetic term gives,
Next, defining the center-of-mass coordinate as $\vec{R}=\frac{1}{A}\sum_{i=1}^{A}\vec{r}_{i}$ allows for the separation of the potential term,
\begin{equation}
  \Ham_{cm}\left( \tilde{\omega} \right) = \Top_{\text{cm}} + \frac{1}{2}mA\tilde{\omega}^{2}\Rop_{\text{cm}}^{2} - \frac{3}{2}\hbar\tilde{\omega} = \frac{1}{2mA}\sum_{ij}^{A}\mathbf{p}}_{i}\cdot\mathbf{p}}_{j} + \frac{m\tilde{\omega}^{2}}{2A}\sum_{ij}^{A}\mathbf{r}_{i}\cdot\mathbf{r}_{j} - \frac{3}{2}\hbar\tilde{\omega}
\end{equation}
  
Use of Jacobi coordinates \cite{BISHOP19901341,NOGGA2002054003}

\begin{equation}
  E_{\text{cm}}\left( \omega \right) \equiv \left\langle \Ham_{cm}\left( \tilde{\omega} \right) \right\rangle \simeq \frac{1}{2\lambda}\left( \left\langle \Ham_{\text{in}} + \lambda\Ham_{cm}\left( \tilde{\omega} \right) \right\rangle - \left\langle \Ham_{\text{in}} - \lambda\Ham_{cm}\left( \tilde{\omega} \right) \right\rangle \right)
\end{equation}

\cite{HAGEN2009062503}
require that $E_{\text{cm}}\left( \tilde{\omega} \right) = 0$ and use $\left\langle \Top_{\text{cm}} \right\rangle = \frac{3}{4}\hbar\tilde{\omega}$
\begin{gather}
  \Ham_{cm}\left( \omega \right) + \frac{3}{2}\hbar\omega - \Top_{\text{cm}} = \frac{\omega^{2}}{\tilde{\omega}^{2}}\left( \Ham_{cm}\left( \tilde{\omega} \right)  + \frac{3}{2}\hbar\tilde{\omega} - \Top_{\text{cm}} \right) \notag \\
  \hbar\tilde{\omega} = \hbar\omega + \frac{2}{3}E_{\text{cm}}\left( \omega \right) \pm \sqrt{\frac{4}{9}\left( E_{\text{cm}}\left( \omega \right) \right)^{2} + \frac{4}{3}\hbar\omegaE_{\text{cm}}\left( \omega \right)}
\end{gather}

Lawson-Gloeckner method, \cite{GLOECKNER1974313}.
\begin{equation} \label{eq:lawsongloeckner}
  \Ham^{\hspace{1pt}}_{\text{in}} = \Ham_{\text{in}} + \beta\Ham_{\text{cm}}\left( \tilde{\omega} \right)
\end{equation}


\end{document}
