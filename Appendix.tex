\documentclass[thesis.tex]{subfiles}

\begin{document}

\chapter{CCSD Diagrams} \label{chapter:ccsd_appendix}

The following diagrams and their corresponding algebraic expressions comprise the different contributions to the CCSD cluster amplitudes \textit{without} directly building the effective Hamiltonian, $\EHam$.  The boxed diagrams are automatically zero in a Hartree-Fock basis.  The contributions to the CCSD singles equation, \eqref{eq:ccsd2}, are given by Eqs.\ \eqref{eq:ccsd_t1_1}-\eqref{eq:ccsd_t1_8}.
\begin{align} \label{eq:ccsd_t1_1}
  \hat{f}_{N}\hat{t}_{1}|\Phi_{0}\rangle_{c}\ &= \ddiagram{CCSD_t1/CCSD_t1-figure0} + \diagram{CCSD_t1/CCSD_t1-figure1} + \diagram{CCSD_t1/CCSD_t1-figure2} \notag \\
  &= \dboxed{\fint{a}{i}} - \sum\limits_{\mathclap{k}}\fint{k}{i}\tamp{a}{k} + \sum\limits_{\mathclap{c}}\fint{a}{c}\tamp{c}{i}
\end{align}
\begin{align} \label{eq:ccsd_t1_2}
  \hat{V}_{N}\hat{t}_{1}|\Phi_{0}\rangle_{c}\ &= \diagram{CCSD_t1/CCSD_t1-figure3} \notag \\
  &= -\sum\limits_{\mathclap{kc}}\vint{ka}{ic}\tamp{c}{k}
\end{align}
\begin{align} \label{eq:ccsd_t1_3}
  \hat{f}_{N}\hat{t}_{2}|\Phi_{0}\rangle_{c}\ &= \ddiagram{CCSD_t1/CCSD_t1-figure4} \notag \\
  &= \dboxed{\sum\limits_{\mathclap{kc}}\fint{k}{c}\tamp{ac}{ki}}
\end{align}
\begin{align} \label{eq:ccsd_t1_4}
  \hat{V}_{N}\hat{t}_{2}|\Phi_{0}\rangle_{c}\ &= \diagram{CCSD_t1/CCSD_t1-figure5} + \diagram{CCSD_t1/CCSD_t1-figure6} \notag \\
  &= \frac{1}{2}\sum\limits_{\mathclap{kcd}}\vint{ka}{cd}\tamp{cd}{ki} + \frac{1}{2}\sum\limits_{\mathclap{klc}}\vint{kl}{ic}\tamp{ca}{kl}
\end{align}
\begin{align} \label{eq:ccsd_t1_5}
  \hat{f}_{N}\hat{t}_{1}^{2}|\Phi_{0}\rangle_{c}\ &= \ddiagram{CCSD_t1/CCSD_t1-figure7} \notag \\
  &= \dboxed{\sum\limits_{\mathclap{kcd}}\fint{l}{d}\tamp{a}{l}\tamp{d}{i}}
\end{align}
\begin{align} \label{eq:ccsd_t1_6}
  \hat{V}_{N}\hat{t}_{1}^{2}|\Phi_{0}\rangle_{c}\ &= \diagram{CCSD_t1/CCSD_t1-figure8} + \diagram{CCSD_t1/CCSD_t1-figure9} \notag \\
  &= \sum\limits_{\mathclap{kcd}}\vint{ka}{cd}\tamp{c}{k}\tamp{d}{i} + \sum\limits_{\mathclap{klc}}\vint{kl}{ic}\tamp{c}{k}\tamp{a}{l}
\end{align}
\begin{align} \label{eq:ccsd_t1_7}
  \hat{V}_{N}\hat{t}_{1}\hat{\text{t}}_{2}|\Phi_{0}\rangle_{c}\ &= \diagram{CCSD_t1/CCSD_t1-figure10} + \diagram{CCSD_t1/CCSD_t1-figure11} + \diagram{CCSD_t1/CCSD_t1-figure12} \notag \\
  &= -\frac{1}{2}\sum\limits_{\mathclap{klcd}}\vint{kl}{cd}\tamp{cd}{ki}\tamp{a}{l} - \frac{1}{2}\sum\limits_{\mathclap{klcd}}\vint{kl}{cd}\tamp{ca}{kl}\tamp{d}{i} + \sum\limits_{\mathclap{klcd}}\vint{kl}{cd}\tamp{ad}{il}\tamp{c}{k}
\end{align}
\begin{align} \label{eq:ccsd_t1_8}
  \hat{V}_{N}\hat{t}_{1}^{3}|\Phi_{0}\rangle_{c}\ &= \diagram{CCSD_t1/CCSD_t1-figure13} \notag \\
  &= -\sum\limits_{\mathclap{klcd}}\vint{kl}{cd}\tamp{c}{k}\tamp{d}{i}\tamp{a}{l}
\end{align}

%%%%%%%%%%%%%%%%%%%%%%%%%%%%%%%%%%%%%%%%%%%%%%%%%%%%%%%%%%%%%%%%%%%%%%%%%%%%%%%%%%%

The contributions to the CCSD doubles equation, \eqref{eq:ccsd2}, are given by Eqs.\ \eqref{eq:ccsd_t2_1}-\eqref{eq:ccsd_t2_11}.
\begin{align} \label{eq:ccsd_t2_1}
  \hat{V}_{N}|\Phi_{0}\rangle_{c}\ &= \diagram{CCSD_t2/CCSD_t2-figure0} \notag \\
  &= \vint{ab}{ij}
\end{align}
\begin{align} \label{eq:ccsd_t2_2}
  \hat{f}_{N}\hat{t}_{2}|\Phi_{0}\rangle_{c}\ &= \diagram{CCSD_t2/CCSD_t2-figure1} + \diagram{CCSD_t2/CCSD_t2-figure2} \notag \\
  &= \Perm{ab}\sum\limits_{\mathclap{c}}\fint{b}{c}\tamp{ac}{ij} - \Perm{ij}\sum\limits_{\mathclap{k}}\fint{k}{j}\tamp{ab}{ik}
\end{align}
\begin{align} \label{eq:ccsd_t2_3}
  \hat{V}_{N}\hat{t}_{1}|\Phi_{0}\rangle_{c}\ &= \diagram{CCSD_t2/CCSD_t2-figure3} + \diagram{CCSD_t2/CCSD_t2-figure4} \notag \\
  &= -\Perm{ab}\sum\limits_{\mathclap{k}}\vint{kb}{ij}\tamp{a}{k} + \Perm{ij}\sum\limits_{\mathclap{c}}\vint{ab}{cj}\tamp{c}{i}
\end{align}
\begin{align} \label{eq:ccsd_t2_4}
  \hat{V}_{N}\hat{t}_{2}|\Phi_{0}\rangle_{c}\ &= \diagram{CCSD_t2/CCSD_t2-figure5} + \diagram{CCSD_t2/CCSD_t2-figure6} + \diagram{CCSD_t2/CCSD_t2-figure7} \notag \\
  &= \frac{1}{2}\sum\limits_{\mathclap{kl}}\vint{kl}{ij}\tamp{ab}{kl} + \frac{1}{2}\sum\limits_{\mathclap{cd}}\vint{ab}{cd}\tamp{cd}{ij} - \Perm{ij|ab}\sum\limits_{\mathclap{kc}}\vint{kb}{ic}\tamp{ac}{kj}
\end{align}
\begin{align} \label{eq:ccsd_t2_5}
  \hat{V}_{N}\hat{t}_{1}^{2}|\Phi_{0}\rangle_{c}\ &= \diagram{CCSD_t2/CCSD_t2-figure8} + \diagram{CCSD_t2/CCSD_t2-figure9} + \diagram{CCSD_t2/CCSD_t2-figure10} \notag \\
  &= \sum\limits_{\mathclap{kl}}\vint{kl}{ij}\tamp{a}{k}\tamp{b}{l} + \sum\limits_{\mathclap{cd}}\vint{ab}{cd}\tamp{c}{i}\tamp{d}{j} - \Perm{ij|ab}\sum\limits_{\mathclap{kc}}\vint{kb}{ic}\tamp{a}{k}\tamp{c}{j}
\end{align}
\begin{align} \label{eq:ccsd_t2_6}
  \hat{V}_{N}\hat{t}_{2}^{2}|\Phi_{0}\rangle_{c}\ &= \diagram{CCSD_t2/CCSD_t2-figure11} + \diagram{CCSD_t2/CCSD_t2-figure12} + \diagram{CCSD_t2/CCSD_t2-figure13} \notag \\
  &+ \diagram{CCSD_t2/CCSD_t2-figure14} \notag \\
  &= \frac{1}{4}\sum\limits_{\mathclap{klcd}}\vint{kl}{cd}\tamp{ab}{kl}\tamp{cd}{ij} + \Perm{ab}\sum\limits_{\mathclap{klcd}}\vint{kl}{cd}\tamp{ac}{lj}\tamp{bd}{ki} - \Perm{ij}\frac{1}{2}\sum\limits_{\mathclap{klcd}}\vint{kl}{cd}\tamp{ab}{lj}\tamp{cd}{ki} \notag \\
  &- \Perm{ab}\frac{1}{2}\sum\limits_{\mathclap{klcd}}\vint{kl}{cd}\tamp{db}{ij}\tamp{ca}{kl}
\end{align}
\begin{align} \label{eq:ccsd_t2_7}
  \hat{f}_{N}\hat{t}_{1}\hat{t}_{2}|\Phi_{0}\rangle_{c}\ &= \ddiagram{CCSD_t2/CCSD_t2-figure15} + \ddiagram{CCSD_t2/CCSD_t2-figure16} \notag \\
  &= -\dboxed{\Perm{ab}\sum\limits_{\mathclap{kc}}\fint{k}{c}\tamp{a}{k}\tamp{cb}{ij}} - \dboxed{\Perm{ij}\sum\limits_{\mathclap{kc}}\fint{k}{c}\tamp{c}{i}\tamp{ab}{kj}}
\end{align}
\begin{align} \label{eq:ccsd_t2_8}
  \hat{V}_{N}\hat{t}_{1}\hat{t}_{2}|\Phi_{0}\rangle_{c}\ &= \diagram{CCSD_t2/CCSD_t2-figure17} + \diagram{CCSD_t2/CCSD_t2-figure18} + \diagram{CCSD_t2/CCSD_t2-figure19} \notag \\
  &+ \diagram{CCSD_t2/CCSD_t2-figure20} + \diagram{CCSD_t2/CCSD_t2-figure21} + \diagram{CCSD_t2/CCSD_t2-figure22} \notag \\
  &= \Perm{ij|ab}\sum\limits_{\mathclap{kcd}}\vint{ka}{cd}\tamp{bc}{jk}\tamp{d}{i} - \Perm{ij|ab}\sum\limits_{\mathclap{klc}}\vint{kl}{ci}\tamp{bc}{jk}\tamp{a}{l} - \Perm{ab}\frac{1}{2}\sum\limits_{\mathclap{kcd}}\vint{kb}{cd}\tamp{cd}{ij}\tamp{a}{k} \notag \\
  &+ \Perm{ij}\frac{1}{2}\sum\limits_{\mathclap{klc}}\vint{kl}{cj}\tamp{ab}{kl}\tamp{c}{i} + \Perm{ab}\sum\limits_{\mathclap{kcd}}\vint{ka}{cd}\tamp{c}{k}\tamp{db}{ij} - \Perm{ij}\sum\limits_{\mathclap{klc}}\vint{kl}{ci}\tamp{c}{k}\tamp{ab}{lj}
\end{align}
\begin{align} \label{eq:ccsd_t2_9}
  \hat{V}_{N}\hat{t}_{1}^{3}|\Phi_{0}\rangle_{c}\ &= \diagram{CCSD_t2/CCSD_t2-figure23} + \diagram{CCSD_t2/CCSD_t2-figure24} \notag \\
  &= -\Perm{ij|ab}\sum\limits_{\mathclap{kcd}}\vint{kb}{cd}\tamp{a}{k}\tamp{c}{i}\tamp{d}{j} + \Perm{ij|ab}\sum\limits_{\mathclap{klc}}\vint{kl}{cj}\tamp{c}{i}\tamp{a}{k}\tamp{b}{l}
\end{align}
\begin{align} \label{eq:ccsd_t2_10}
  \hat{V}_{N}\hat{t}_{1}^{2}\hat{t}_{2}|\Phi_{0}\rangle_{c}\ &= \diagram{CCSD_t2/CCSD_t2-figure25} + \diagram{CCSD_t2/CCSD_t2-figure26} + \diagram{CCSD_t2/CCSD_t2-figure27} \notag \\
  &+ \diagram{CCSD_t2/CCSD_t2-figure28} + \diagram{CCSD_t2/CCSD_t2-figure29} \notag \\
  &= \frac{1}{2}\sum\limits_{\mathclap{klcd}}\vint{kl}{cd}\tamp{ab}{kl}\tamp{c}{i}\tamp{d}{j} + \frac{1}{2}\sum\limits_{\mathclap{klcd}}\vint{kl}{cd}\tamp{cd}{ij}\tamp{a}{k}\tamp{b}{l} + \Perm{ij|ab}\sum\limits_{\mathclap{klcd}}\vint{kl}{cd}\tamp{ac}{lj}\tamp{b}{k}\tamp{d}{i} \notag \\
  &- \Perm{ij}\frac{1}{2}\sum\limits_{\mathclap{klcd}}\vint{kl}{cd}\tamp{ab}{lj}\tamp{c}{k}\tamp{d}{i} - \Perm{ab}\frac{1}{2}\sum\limits_{\mathclap{klcd}}\vint{kl}{cd}\tamp{db}{ij}\tamp{c}{k}\tamp{a}{l}
\end{align}
\begin{align} \label{eq:ccsd_t2_11}
  \hat{V}_{N}\hat{t}_{1}^{4}|\Phi_{0}\rangle_{c}\ &= \diagram{CCSD_t2/CCSD_t2-figure30} \notag \\
  &= \sum\limits_{\mathclap{klcd}}\vint{kl}{cd}\tamp{a}{k}\tamp{b}{l}\tamp{c}{i}\tamp{d}{j}
\end{align}




\chapter{Effective Hamiltonian Diagrams} \label{chapter:eff_ham_diagrams}

The following diagrams and their corresponding algebraic expressions comprise the different components of the CCSD effective Hamiltonian, $\EHam$.  Because some components are used as intermediates to build other components, they must be built in the order written.  Some intermediate components overcount some diagrams which motivates the need for further intermediates, denoted by $\xxint{}{}, \xxxint{}{}$, and $\xxxxint{}{}$.  The boxed diagrams are automatically zero in a Hartree-Fock basis.  The one-body components to the CCSD effective Hamiltonian, Eq.\ \eqref{eq:cc_heff1}, are given by Eqs.\ \eqref{eq:eff1}-\eqref{eq:eff4}.
\begin{align} \label{eq:eff1}
  \diagram{CCSD_1b/CCSD_1b-figure0} &= \ddiagram{CCSD_1b/CCSD_1b-figure1} + \diagram{CCSD_1b/CCSD_1b-figure2} \notag \\
  \xint{i}{a} &= \dboxed{\fint{i}{a}} + \sum\limits_{kc}\vint{ik}{ac}\tamp{c}{k}
\end{align}
\begin{align} \label{eq:eff2}
  \diagram{CCSD_1b/CCSD_1b-figure3} &= \diagram{CCSD_1b/CCSD_1b-figure4} + \diagram{CCSD_1b/CCSD_1b-figure5} + \diagram{CCSD_1b/CCSD_1b-figure6} + \diagram{CCSD_1b/CCSD_1b-figure7} \notag \\
  \xint{a}{b} &= \fint{a}{b} - \frac{1}{2}\sum\limits_{klc}\vint{kl}{bc}\tamp{ac}{kl} + \sum\limits_{kc}\vint{ka}{cb}\tamp{c}{k} - \sum\limits_{k}\xint{k}{b}\tamp{a}{k}
\end{align}
\begin{align} \label{eq:eff3}
  \diagram{CCSD_1b/CCSD_1b-figure8} &= \diagram{CCSD_1b/CCSD_1b-figure9} + \diagram{CCSD_1b/CCSD_1b-figure10} + \diagram{CCSD_1b/CCSD_1b-figure11} \notag \\
  \xxint{i}{j} &= \fint{i}{j} + \frac{1}{2}\sum\limits_{kcd}\vint{ik}{cd}\tamp{cd}{jk} + \sum\limits_{kc}\vint{ik}{jc}\tamp{c}{k}
\end{align}
\begin{align} \label{eq:eff4}
  \diagram{CCSD_1b/CCSD_1b-figure12} &= \diagram{CCSD_1b/CCSD_1b-figure13} + \diagram{CCSD_1b/CCSD_1b-figure14} \notag \\
  \xint{i}{j} &= \xxint{i}{j} + \sum\limits_{c}\xint{i}{c}\tamp{c}{j}
\end{align}

Once the one-body components have been constructed, the pseudo-linear form for the CCSD singles equation, Eq.\ \eqref{eq:ccsd1}, can be evaluated,
\begin{align} \label{eq:singles_linear}
  \diagram{CCSD_1b/CCSD_1b-figure15} = 0 &= \ddiagram{CCSD_1b/CCSD_1b-figure16} + \diagram{CCSD_1b/CCSD_1b-figure17} + \diagram{CCSD_1b/CCSD_1b-figure18} + \diagram{CCSD_1b/CCSD_1b-figure19} \notag \\[-1.5ex]
  &+ \diagram{CCSD_1b/CCSD_1b-figure20} + \diagram{CCSD_1b/CCSD_1b-figure21} + \diagram{CCSD_1b/CCSD_1b-figure22} \notag \\
  \xint{a}{i} = 0 &= \dboxed{\fint{a}{i}} + \sum\limits_{\mathclap{c}}\xint{a}{c}\tamp{c}{i} - \sum\limits_{\mathclap{k}}\xxint{k}{i}\tamp{a}{k} + \sum\limits_{\mathclap{kc}}\vint{ka}{ci}\tamp{c}{k} \notag \\
  &+ \frac{1}{2}\sum\limits_{\mathclap{kcd}}\vint{ka}{cd}\tamp{cd}{ki} - \frac{1}{2}\sum\limits_{\mathclap{klc}}\vint{kl}{ic}\tamp{ac}{kl} + \sum\limits_{\mathclap{kc}}\xint{k}{c}\tamp{ac}{ik}.
\end{align}
During a CC iteration, it's possible to use these updated singles amplitues $\Top_{1}$ when evaluating the doubles amplitudes $\Top_{2}$ in the same iteration, which can accelerate the convergence.

The two-body components to the CCSD effective Hamiltonian, Eq.\ \eqref{eq:cc_heff1}, are given by Eqs.\ \eqref{eq:eff5}-\eqref{eq:eff19}.
\begin{align} \label{eq:eff5}
  \diagram{CCSD_2b/CCSD_2b-figure0} &= \diagram{CCSD_2b/CCSD_2b-figure1} + \frac{1}{2}\fdiagram{CCSD_2b/CCSD_2b-figure2} \notag \\
  \xxint{ia}{bc} &= \vint{ia}{bc} - \frac{1}{2}\sum\limits_{k}\vint{ik}{bc}\tamp{a}{k}
\end{align}
\begin{align} \label{eq:eff6}
  \diagram{CCSD_2b/CCSD_2b-figure3} &= \diagram{CCSD_2b/CCSD_2b-figure4} + \diagram{CCSD_2b/CCSD_2b-figure5} \notag \\
  \xint{ia}{bc} &= \vint{ia}{bc} - \sum\limits_{k}\vint{ik}{bc}\tamp{a}{k}
\end{align}
\begin{align} \label{eq:eff7}
  \diagram{CCSD_2b/CCSD_2b-figure6} &= \diagram{CCSD_2b/CCSD_2b-figure7} + \frac{1}{2}\fdiagram{CCSD_2b/CCSD_2b-figure8} \notag \\
  \xxint{ij}{ka} &= \vint{ij}{ka} + \frac{1}{2}\sum\limits_{c}\vint{ij}{ca}\tamp{c}{k}
\end{align}
\begin{align} \label{eq:eff8}
  \diagram{CCSD_2b/CCSD_2b-figure9} &= \diagram{CCSD_2b/CCSD_2b-figure10} + \diagram{CCSD_2b/CCSD_2b-figure11} \notag \\
  \xint{ij}{ka} &= \vint{ij}{ka} + \sum\limits_{c}\vint{ij}{ca}\tamp{c}{k}
\end{align}
\begin{align} \label{eq:eff9}
  \diagram{CCSD_2b/CCSD_2b-figure12} &= \diagram{CCSD_2b/CCSD_2b-figure13} + \diagram{CCSD_2b/CCSD_2b-figure14} \notag \\
  \xxint{ab}{cd} &= \vint{ab}{cd} - \Perm{ab}\sum\limits_{k}\xxint{kb}{cd}\tamp{a}{k}
\end{align}
\begin{align} \label{eq:eff10}
  \diagram{CCSD_2b/CCSD_2b-figure15} &= \diagram{CCSD_2b/CCSD_2b-figure16} + \diagram{CCSD_2b/CCSD_2b-figure17} \notag \\
  \xint{ab}{cd} &= \xxint{ab}{cd} + \frac{1}{2}\sum\limits_{kl}\vint{kl}{cd}\tamp{ab}{kl}
\end{align}
\begin{align} \label{eq:eff11}
  \diagram{CCSD_2b/CCSD_2b-figure18} &= \diagram{CCSD_2b/CCSD_2b-figure19} + \diagram{CCSD_2b/CCSD_2b-figure20} + \diagram{CCSD_2b/CCSD_2b-figure21} \notag \\
  \xint{ij}{kl} &= \vint{ij}{kl} + \frac{1}{2}\sum\limits_{cd}\vint{ij}{cd}\tamp{cd}{kl} + \Perm{kl}\sum\limits_{c}\xxint{ij}{kc}\tamp{c}{l}
\end{align}
\begin{align} \label{eq:eff12}
  \diagram{CCSD_2b/CCSD_2b-figure22} &= \diagram{CCSD_2b/CCSD_2b-figure23} + \diagram{CCSD_2b/CCSD_2b-figure24} + \frac{1}{2}\fdiagram{CCSD_2b/CCSD_2b-figure25} \notag \\
  \xxint{ia}{jb} &= \vint{ia}{jb} + \sum\limits_{c}\xxint{ia}{cb}\tamp{c}{j} - \frac{1}{2}\sum\limits_{k}\vint{ik}{jb}\tamp{a}{k}
\end{align}
\begin{align} \label{eq:eff13}
  \diagram{CCSD_2b/CCSD_2b-figure26} &= \diagram{CCSD_2b/CCSD_2b-figure27} + \frac{1}{2}\fdiagram{CCSD_2b/CCSD_2b-figure28} + \frac{1}{2}\fdiagram{CCSD_2b/CCSD_2b-figure29} \notag \\
  \xxxint{ia}{jb} &= \vint{ia}{jb} + \frac{1}{2}\sum\limits_{c}\xxint{ia}{cb}\tamp{c}{j} - \frac{1}{2}\sum\limits_{k}\vint{ik}{jb}\tamp{a}{k}
\end{align}
\begin{align} \label{eq:eff14}
  \diagram{CCSD_2b/CCSD_2b-figure30} &= \diagram{CCSD_2b/CCSD_2b-figure31} + \frac{1}{2}\fdiagram{CCSD_2b/CCSD_2b-figure32} + \diagram{CCSD_2b/CCSD_2b-figure33} \notag \\
  \xxxxint{ia}{jb} &= \vint{ia}{jb} + \frac{1}{2}\sum\limits_{c}\xint{ia}{cb}\tamp{c}{j} - \sum\limits_{k}\vint{ik}{jb}\tamp{a}{k}
\end{align}
\begin{align} \label{eq:eff15}
  \diagram{CCSD_2b/CCSD_2b-figure34} &= \diagram{CCSD_2b/CCSD_2b-figure35} + \left(\frac{1}{2}\hspace{-0.5mm}\right)\fdiagram{CCSD_2b/CCSD_2b-figure36} + \frac{1}{2}\fdiagram{CCSD_2b/CCSD_2b-figure37} \notag \\
  \xint{ia}{jb} &= \xxxxint{ia}{jb} - \left(\frac{1}{2}\right)\sum\limits_{kc}\vint{ik}{cb}\tamp{ca}{jk} + \frac{1}{2}\sum\limits_{c}\xint{ia}{cb}\tamp{c}{j}
\end{align}
The factor of $\left(\frac{1}{2}\right)$ is applied when solving the CCSD equations but omitted when applying the effective Hamiltonian in post-CC methods.
\begin{align} \label{eq:eff16}
  \diagram{CCSD_2b/CCSD_2b-figure38} &= \diagram{CCSD_2b/CCSD_2b-figure39} + \frac{1}{2}\fdiagram{CCSD_2b/CCSD_2b-figure40} + \diagram{CCSD_2b/CCSD_2b-figure41} \notag \\
  \xxint{ab}{ic} &= \vint{ab}{ic} + \frac{1}{2}\sum\limits_{d}\vint{ab}{dc}\tamp{d}{i} - \Perm{ab}\sum\limits_{k}\xxxint{kb}{ic}\tamp{a}{k}
\end{align}
\begin{align} \label{eq:eff17}
  \diagram{CCSD_2b/CCSD_2b-figure42} &= \diagram{CCSD_2b/CCSD_2b-figure43} + \diagram{CCSD_2b/CCSD_2b-figure44} + \diagram{CCSD_2b/CCSD_2b-figure45} \notag \\[-1.5ex]
  &+ \diagram{CCSD_2b/CCSD_2b-figure46} + \diagram{CCSD_2b/CCSD_2b-figure47} + \diagram{CCSD_2b/CCSD_2b-figure48} \notag \\
  \xint{ab}{ic} &= \vint{ab}{ic} + \sum\limits_{d}\vint{ab}{dc}\tamp{d}{i} - \Perm{ab}\sum\limits_{k}\xxint{kb}{ic}\tamp{a}{k} \notag \\
  &- \sum\limits_{k}\xint{k}{c}\tamp{ab}{ik} + \Perm{ab}\sum\limits_{kd}\xint{kb}{dc}\tamp{ad}{ik} + \frac{1}{2}\sum\limits_{kl}\xint{kl}{ic}\tamp{ab}{kl}
\end{align}
\begin{align} \label{eq:eff18}
  \diagram{CCSD_2b/CCSD_2b-figure49} &= \diagram{CCSD_2b/CCSD_2b-figure50} + \frac{1}{2}\fdiagram{CCSD_2b/CCSD_2b-figure51} \notag \\
  \xxint{ia}{jk} &= \vint{ia}{jk} - \frac{1}{2}\sum\limits_{l}\vint{il}{jk}\tamp{a}{l}
\end{align}
\begin{align} \label{eq:eff19}
  \diagram{CCSD_2b/CCSD_2b-figure52} &= \diagram{CCSD_2b/CCSD_2b-figure53} + \diagram{CCSD_2b/CCSD_2b-figure54} + \diagram{CCSD_2b/CCSD_2b-figure55} \notag \\[-1.5ex]
  &+ \diagram{CCSD_2b/CCSD_2b-figure56} + \diagram{CCSD_2b/CCSD_2b-figure57} + \diagram{CCSD_2b/CCSD_2b-figure58} \notag \\
  \xint{ia}{jk} &= \vint{ia}{jk} - \sum\limits_{l}\vint{il}{jk}\tamp{a}{l} + \Perm{jk}\sum\limits_{c}\xxxxint{ia}{jc}\tamp{c}{k} \notag \\
  &+ \Perm{jk}\sum\limits_{lc}\xint{il}{jc}\tamp{ca}{lk} + \frac{1}{2}\sum\limits_{cd}\xint{ia}{cd}\tamp{cd}{jk} + \sum\limits_{c}\xint{i}{c}\tamp{ca}{jk}
\end{align}


\begin{align} \label{eq:double_linear}
  \diagram{CCSD_2b/CCSD_2b-figure59} = 0 &= \diagram{CCSD_2b/CCSD_2b-figure60} + \diagram{CCSD_2b/CCSD_2b-figure61} + \diagram{CCSD_2b/CCSD_2b-figure62} \notag \\[-1.5ex]
  &\hspace{-100pt}+ \diagram{CCSD_2b/CCSD_2b-figure63} + \diagram{CCSD_2b/CCSD_2b-figure64} + \diagram{CCSD_2b/CCSD_2b-figure65} + \diagram{CCSD_2b/CCSD_2b-figure66} + \diagram{CCSD_2b/CCSD_2b-figure67} \notag \\
  \xint{ab}{ij} = 0 &= \vint{ab}{ij} + \Perm{ab}\sum\limits_{\mathclap{c}}\xint{a}{c}\tamp{cb}{ij} - \Perm{ij}\sum\limits_{\mathclap{k}}\xint{k}{i}\tamp{ab}{kj} \notag \\
  &\hspace{-100pt}+ \frac{1}{2}\sum\limits_{\mathclap{cd}}\xxint{ab}{cd}\tamp{cd}{ij} + \frac{1}{2}\sum\limits_{\mathclap{kl}}\xint{kl}{ij}\tamp{ab}{kl} - \Perm{ab|ij}\sum\limits_{\mathclap{kc}}\xint{kb}{ic}\tamp{ac}{kj} - \Perm{ab}\sum\limits_{\mathclap{k}}\xxint{kb}{ij}\tamp{a}{k} + \Perm{ij}\sum\limits_{\mathclap{c}}\xxint{ab}{ic}\tamp{c}{j}
\end{align}



\chapter{Computational Implementation} \label{chapter:appendix_computational}

The sums involved in building the CC effective Hamiltonian, solving the CC equations, solving the EOM-CC equations, and building effective operators can all be reformulated as matrix-matrix multiplications and thus performed with efficient LAPACK and BLAS routines.  To take advantage of this efficiency, the various cluster amplitudes and matrix elements must be grouped into structures with similar index organization so that summed indices map to the same states and matrix elements.  An additional benefit to these structures is that angular-momentum-coupling coefficients are automatically removed by summing over Clebsch-Gordon coefficients, see chapter \ref{chapter:angular_momentum}.

\section{Symmetry Channels}

Each matrix structure is separated into different symmetry channels for different perutations of its indices.  The channels are denoted as $\Sigma_{\vec{\xi}}$, where $\vec{\xi}$ represents the relevant quantum numbers of a certain channel.  There are four different channel types that are relevant for the structures used in this work.

The direct two-body channel categorizes the vector sum of two single-particle-state quantum numbers, $\Sigma_{\vec{\xi}_{1}}$,
\begin{equation}
  \vec{\xi}_{pq} = \vec{\xi}_{p} + \vec{\xi}_{q}\ \ \longrightarrow\ \ \ket{pq} \in \Sigma_{\vec{\xi}_{1}=\vec{\xi}_{pq}}.
\end{equation}
The cross two-body channel categorizes the vector difference of two single-particle-state quantum numbers or, equivalently, the vector sum of a the quantum numbers of a single-particle state and a time-reversed single-particle state, $\Sigma_{\vec{\xi}_{2}}$,
\begin{equation}
  \vec{\xi}_{p\bar{q}} = \vec{\xi}_{p} - \vec{\xi}_{q} = \vec{\xi}_{p} + \vec{\xi}_{\bar{q}}\ \ \longrightarrow\ \ \ket{p\bar{q}} \in \Sigma_{\vec{\xi}_{2}}.
\end{equation}
The one-body channel categorizes single-particle states by their vector quantum numbers, $\Sigma_{\vec{\xi}_{3}}$,
\begin{equation}
  \vec{\xi}_{p}\ \ \longrightarrow\ \ \ket{p} \in \Sigma_{\vec{\xi}_{3}}.
\end{equation}
The cross three-body state categorizes the vector difference between the quantum numbers of a direct two-body state and a single-particle state or, equivalently, the vector sum of the quantum numbers of a two-body direct state and a time-reversed single-particle state, $\Sigma_{\vec{\xi}_{3}}$,
\begin{equation}
  \vec{\xi}_{pq\bar{r}} = \vec{\xi}_{p} + \vec{\xi}_{q} - \vec{\xi}_{r} = \vec{\xi}_{p} + \vec{\xi}_{q} + \vec{\xi}_{\bar{r}}\ \ \longrightarrow\ \ \ket{pq\bar{r}} \in \Sigma_{\vec{\xi}_{3}=\vec{\xi}_{pq\bar{r}}}.
\end{gather}

\section{Channel-Partitioned Structures} \label{section:channel_matrices}

Different matrix structures are indexed by their channel type: $1$ for direct channels, $2$ for cross channels, and $3$ for one-/three-body channels.  For matrices with more than one structure of the same type, there is an additional index that depends on the specific permutation involved.

For a one-body operator $A^{p}_{q}\normord{\co{p}\ao{q}}$, there is a direct-channel matrix element and a cross-channel matrix element,
\begin{equation}
  \mathbf{A}_{1} = A^{p}_{q}, \hspace{1cm} \mathbf{A}_{2} = A^{p\bar{q}}.
\end{equation}

For a two-body operator $A^{pq}_{rs}\normord{\co{p}\co{q}\ao{s}\ao{r}}$, there is a direct-channel matrix element, four cross-channel matrix elements, and four one-channel matrix elements,
\begin{gather}
  \mathbf{A}_{1} = A^{pq}_{rs}, \notag \\
  \mathbf{A}_{2_{1}} = A^{p\bar{s}}_{r\bar{q}}, \hspace{0.5cm} \mathbf{A}_{2_{2}} = A^{q\bar{r}}_{s\bar{p}}, \hspace{0.5cm} \mathbf{A}_{2_{3}} = A^{p\bar{r}}_{s\bar{q}}, \hspace{0.5cm} \mathbf{A}_{2_{4}} = A^{q\bar{s}}_{r\bar{p}}, \notag \\
  \mathbf{A}_{3_{1}} = A^{p}_{rs\bar{q}}, \hspace{0.5cm} \mathbf{A}_{3_{2}} = A^{q}_{rs\bar{p}}, \hspace{0.5cm} \mathbf{A}_{3_{3}} = A^{pq\bar{s}}_{r}, \hspace{0.5cm} \mathbf{A}_{3_{4}} = A^{pq\bar{r}}_{s}.
\end{gather}

For an EOM operator of the form $A^{pq}_{r}\normord{\co{p}\co{q}\ao{r}}$, there is a direct-channel matrix element, a one-channel matrix element, and two cross-channel matrix elements,
\begin{gather}
  \mathbf{A}_{1} = A^{pq}_{r}, \hspace{1cm} \mathbf{A}_{3} = A^{pq\bar{r}}, \notag \\
  \mathbf{A}_{2_{1}} = A^{p}_{r\bar{q}}, \hspace{1.0cm} \mathbf{A}_{2_{2}} = A^{q}_{r\bar{p}}.
\end{gather}

EOM operators of the form $A^{p}_{qr}\normord{\co{p}\ao{r}\ao{q}}$ have similar structures,
\begin{gather}
  \mathbf{A}_{1} = A^{p}_{qr}, \hspace{1cm} \mathbf{A}_{3} = A_{qr\bar{p}}, \notag \\
  \mathbf{A}_{2_{1}} = A^{p\bar{r}}_{q}, \hspace{1.0cm} \mathbf{A}_{2_{2}} = A^{p\bar{q}}_{r}.
\end{gather}

\section{Matrix Form of $\EHam = \left(\HamN\E^{\Top}\right)_{\mathrm{c}}$}

\begin{align}
  \xint{i}{a} &= \dboxed{\fint{i}{a}} + \vint{i\bar{a}}{c\bar{k}}\tamp{c\bar{k}}{} \notag \\
  \bxint{hp}{2} &\longleftarrow \dboxed{\bfint{hp}{2}} + \bvint{hhpp}{2_{3}}\cdot\btamp{}{2}
\end{align}
\begin{align}
  \xint{a}{b} &= \fint{a}{b} - \frac{1}{2}\tamp{a}{kl\bar{c}}\vint{kl\bar{c}}{b} + \vint{a\bar{b}}{c\bar{k}}\tamp{c\bar{k}}{} - \tamp{a}{k}\xint{k}{b} \notag \\
  \bxint{pp}{3} &\longleftarrow -\frac{1}{2}\btamp{}{3_{1}}\cdot\bvint{hhpp}{3_{3}} - \btamp{}{3}\cdot\bxint{hp}{3} \notag \\
  \bxint{pp}{2} &\longleftarrow \bfint{pp}{2} + \bvint{hppp}{2_{4}}\cdot\btamp{}{2}
\end{align}
\begin{align}
  \xxint{i}{j} &= \fint{i}{j} + \frac{1}{2}\vint{i}{cd\bar{k}}\tamp{cd\bar{k}}{j} + \vint{i\bar{j}}{c\bar{k}}\tamp{c\bar{k}}{} \notag \\
  \bxxint{hh}{3} &\longleftarrow \frac{1}{2}\bvint{hhpp}{3_{1}}\cdot\btamp{}{3_{3}} \notag \\
  \bxxint{hh}{2} &\longleftarrow \bfint{hh}{2} + \bvint{hhhp}{2_{3}}\cdot\btamp{}{2}
\end{align}
\begin{align}
  \xint{i}{j} &= \fint{i}{j} + \frac{1}{2}\vint{i}{cd\bar{k}}\tamp{cd\bar{k}}{j} + \vint{i\bar{j}}{c\bar{k}}\tamp{c\bar{k}}{} + \xint{i}{c}\tamp{c}{j} \notag \\
  \bxint{hh}{3} &\longleftarrow \frac{1}{2}\bvint{hhpp}{3_{1}}\cdot\btamp{}{3_{3}} + \bxint{hp}{3}\cdot\btamp{}{3} \notag \\
  \bxxint{hh}{2} &\longleftarrow \bfint{hh}{2} + \bvint{hhhp}{2_{3}}\cdot\btamp{}{2}
\end{align}
\begin{align}
  \xint{a}{i} &= \dboxed{\fint{a}{i}} + \xint{a}{c}\tamp{c}{i} - \tamp{a}{k}\xxint{k}{i} - \vint{a\bar{i}}{c\bar{k}}\tamp{c\bar{k}}{} + \frac{1}{2}\vint{a}{cd\bar{k}}\tamp{cd\bar{k}}{i} - \frac{1}{2}\tamp{a}{kl\bar{c}}\vint{kl\bar{c}}{i} + \tamp{a\bar{i}}{k\bar{c}}\xint{k\bar{c}}{} \notag \\
  \bxint{ph}{3} &\longleftarrow \bxint{pp}{3}\cdot\btamp{}{3} - \btamp{}{3}\cdot\bxxint{hh}{3} + \frac{1}{2}\bvint{hppp}{3_{2}}\cdot\btamp{}{3_{4}} - \frac{1}{2}\btamp{}{3_{1}}\cdot\bvint{hhhp}{3_{3}} \notag \\
  \bxint{ph}{2} &\longleftarrow \dboxed{\bfint{ph}{2}} - \bvint{hphp}{2_{2}}\cdot\btamp{}{2} + \btamp{}{2_{3}}\cdot\bxint{hp}{2}
\end{align}

%%%%%%%%%%%%%%%%%%%%%%%%%%%%%%%%%%%%%%%%%%%%%%%%%%%%%%%%%%%%%%%%%%%%%%%%%%%%%%%%%%%

\begin{align}
  \xxint{ia}{bc} &= \vint{ia}{bc} - \frac{1}{2}\tamp{a}{k}\vint{k}{bc\bar{i}} \notag \\
  \bxxint{hppp}{3_{2}} &\longleftarrow \bvint{hppp}{3_{2}} - \frac{1}{2}\btamp{}{3}\cdot\bvint{hhpp}{3_{2}}
\end{align}
\begin{align}
  \xint{ia}{bc} &= \vint{ia}{bc} - \tamp{a}{k}\vint{k}{bc\bar{i}} \notag \\
  \bxint{hppp}{3_{2}} &\longleftarrow \bvint{hppp}{3_{2}} - \btamp{}{3}\cdot\bvint{hhpp}{3_{2}}
\end{align}
\begin{align}
  \xxint{ij}{ka} &= \vint{ij}{ka} + \frac{1}{2}\vint{ij\bar{a}}{c}\tamp{c}{k} \notag \\
  \bxxint{hhhp}{3_{3}} &\longleftarrow \bvint{ij}{ka} + \frac{1}{2}\bvint{hhpp}{3_{3}}\cdot\btamp{}{3}
\end{align}
\begin{align}
  \xint{ij}{ka} &= \vint{ij}{ka} + \vint{ij\bar{a}}{c}\tamp{c}{k} \notag \\
  \bxint{hhhp}{3_{3}} &\longleftarrow \bvint{ij}{ka} + \bvint{hhpp}{3_{3}}\cdot\btamp{}{3}
\end{align}
\begin{align}
  \xxint{ab}{cd} &= \vint{ab}{cd} - \Perm{ab}\tamp{a}{k}\xxint{k}{cd\bar{b}} \notag \\
  \bxxint{pppp}{1} &\longleftarrow \bvint{pppp}{1} \notag \\
  \bxxint{pppp}{3_{1(2)}} &\longleftarrow \mp\btamp{}{3}\cdot\bxxint{hppp}{3_{1}}
\end{align}
\begin{align}
  \xint{ab}{cd} &= \xxint{ab}{cd} + \frac{1}{2}\tamp{ab}{kl}\vint{kl}{cd} \notag \\
  \bxint{pppp}{1} &\longleftarrow \bxxint{pppp}{1} + \frac{1}{2}\btamp{}{1}\cdot\bvint{hhpp}{1}
\end{align}
\begin{align}
  \xint{ij}{kl} &= \vint{ij}{kl} + \frac{1}{2}\vint{ij}{cd}\tamp{cd}{kl} + \Perm{kl}\xxint{ij\bar{k}}{c}\tamp{c}{l} \notag \\
  \bxint{hhhh}{1} &\longleftarrow \bvint{hhhh}{1} + \frac{1}{2}\bvint{hhpp}{1}\cdot\btamp{}{1} \notag \\
  \bxint{hhhh}{3_{3(4)}} &\longleftarrow \mp\bxxint{hhhp}{3_{4}}\cdot\btamp{}{3}
\end{align}
\begin{align}
  \xxint{ia}{jb} &= \vint{ia}{jb} + \xxint{ia\bar{b}}{c}\tamp{c}{j} - \frac{1}{2}\tamp{a}{k}\vint{k}{jb\bar{i}} \notag \\
  \bxxint{hphp}{2_{1}} &\longleftarrow \bvint{hphp}{2_{1}} \notag \\
  \bxxint{hphp}{3_{3}} &\longleftarrow \bxxint{hppp}{3_{3}}\cdot\btamp{}{3} \notag \\
  \bxxint{hphp}{3_{2}} &\longleftarrow -\frac{1}{2}\btamp{}{3}\cdot\bvint{hhhp}{3_{2}}
\end{align}
\begin{align}
  \xxxint{ia}{jb} &= \vint{ia}{jb} + \frac{1}{2}\xxint{ia\bar{b}}{c}\tamp{c}{j} - \frac{1}{2}\tamp{a}{k}\vint{k}{jb\bar{i}} \notag \\
  \bxxxint{hphp}{2_{1}} &\longleftarrow \bvint{hphp}{2_{1}} \notag \\
  \bxxxint{hphp}{3_{3}} &\longleftarrow \frac{1}{2}\bxxint{hppp}{3_{3}}\cdot\btamp{}{3} \notag \\
  \bxxxint{hphp}{3_{2}} &\longleftarrow -\frac{1}{2}\btamp{}{3}\cdot\bvint{hhhp}{3_{2}}
\end{align}
\begin{align}
  \xxxxint{ia}{jb} &= \vint{ia}{jb} + \frac{1}{2}\xint{ia\bar{b}}{c}\tamp{c}{j} - \tamp{a}{k}\vint{k}{jb\bar{i}} \notag \\
  \bxxxxint{hphp}{2_{1}} &\longleftarrow \bvint{hphp}{2_{1}} \notag \\
  \bxxxxint{hphp}{3_{3}} &\longleftarrow \frac{1}{2}\bxxint{hppp}{3_{3}}\cdot\btamp{}{3} \notag \\
  \bxxxxint{hphp}{3_{2}} &\longleftarrow -\btamp{}{3}\cdot\bvint{hhhp}{3_{2}}
\end{align}
\begin{align}
  \xint{ia}{jb} &= \vint{ia}{jb} + \xint{ia\bar{b}}{c}\tamp{c}{j} - \tamp{a}{k}\vint{k}{jb\bar{i}} - \left(\frac{1}{2}\right)\vint{i\bar{b}}{c\bar{k}}\tamp{c\bar{k}}{j\bar{a}} \notag \\
  \bxint{hphp}{2_{1}} &\longleftarrow \bvint{hphp}{2_{1}} - \left(\frac{1}{2}\right)\bvint{hhpp}{2_{1}}\btamp{}{2_{1}} \notag \\
  \bxint{hphp}{3_{3}} &\longleftarrow \bxint{hppp}{3_{3}}\cdot\btamp{}{3} \notag \\
  \bxint{hphp}{3_{2}} &\longleftarrow -\btamp{}{3}\cdot\bvint{hhhp}{3_{2}}
\end{align}
\begin{align}
  \xxint{ab}{ic} &= \vint{ab}{ic} + \frac{1}{2}\vint{ab\bar{c}}{d}\tamp{d}{i} - \Perm{ab}\tamp{a}{k}\xxxint{k}{ic\bar{b}} \notag \\
  \bxxint{pphp}{3_{3}} &\longleftarrow \bvint{pphp}{3_{3}} + \frac{1}{2}\bvint{pppp}{3_{3}}\cdot\btamp{}{3} \notag \\
  \bxxint{pphp}{3_{1(2)}} &\longleftarrow \mp\btamp{}{3}\cdot\bxxxint{hphp}{3_{1}}
\end{align}
\begin{align}
  \xint{ab}{ic} &= \vint{ab}{ic} + \vint{ab\bar{c}}{d}\btamp{d}{i} - \Perm{ab}\tamp{a}{k}\xxint{k}{ic\bar{b}} - \tamp{ab\bar{i}}{k}\xint{k}{c} + \Perm{ab}\tamp{a\bar{i}}{k\bar{d}}\xint{k\bar{d}}{c\bar{b}} + \frac{1}{2}\tamp{ab}{kl}\xint{kl}{ic} \notag \\
  \bxint{pphp}{3_{3}} &\longleftarrow \bvint{pphp}{3_{3}} + \bvint{pppp}{3_{3}}\cdot\btamp{}{3} \notag \\
  \bxint{pphp}{3_{1(2)}} &\longleftarrow \mp\btamp{}{3}\cdot\bxxint{hphp}{3_{1}} \notag \\
  \bxint{pphp}{3_{4}} &\longleftarrow -\btamp{}{3_{4}}\cdot\bxint{hp}{3} \notag \\
  \bxint{pphp}{2_{2(3)}} &\longleftarrow \mp\btamp{}{2_{3}}\cdot\bxint{hppp}{2_{3}} \notag \\
  \bxint{pphp}{1} &\longleftarrow \frac{1}{2}\btamp{}{1}\cdot\bxint{hhhp}{1}
\end{align}
\begin{align}
  \xxint{ia}{jk} &= \vint{ia}{jk} - \frac{1}{2}\tamp{a}{l}\vint{l}{jk\bar{i}} \notag \\
  \bxxint{hphh}{3_{2}} &\longleftarrow \bvint{hphh}{3_{2}} - \frac{1}{2}\btamp{}{3}\cdot\bvint{hhhh}{3_{2}}
\end{align}
\begin{align}
  \xint{ia}{jk} &= \vint{ia}{jk} - \tamp{a}{l}\vint{l}{jk\bar{i}} + \Perm{jk}\xxxxint{ia\bar{j}}{c}\tamp{c}{k} + \Perm{jk}\xint{i\bar{j}}{c\bar{l}}\tamp{c\bar{l}}{k\bar{a}} + \frac{1}{2}\xint{ia}{cd}\tamp{cd}{jk} + \xint{i}{c}\tamp{c}{jk\bar{a}} \notag \\
  \bxint{hphh}{3_{2}} &\longleftarrow \bvint{hphh}{3_{2}} - \btamp{}{3}\cdot\bvint{hhhh}{3_{2}} \notag \\
  \bxint{hphh}{3_{3(4)}} &\longleftarrow \mp\bxxxxint{hphp}{3_{4}}\cdot\btamp{}{3} \notag \\
  \bxint{hphh}{2_{1(3)}} &\longleftarrow \mp\bxint{hhhp}{2_{3}}\cdot\btamp{}{2_{3}} \notag \\
  \bxint{hphh}{1} &\longleftarrow \frac{1}{2}\bxint{hppp}{1}\cdot\btamp{}{1} \notag \\
  \bxint{hphh}{3_{1}} &\longleftarrow \bxint{hp}{3}\cdot\btamp{}{3_{1}}  
\end{align}
\begin{align}
  \xint{ab}{ij} &= \vint{ab}{ij} + \Perm{ab}\xint{a}{c}\tamp{c}{ij\bar{b}} - \Perm{ij}\tamp{ab\bar{j}}{k}\xint{k}{i} + \frac{1}{2}\xxint{ab}{cd}\tamp{cd}{ij} + \frac{1}{2}\tamp{ab}{kl}\xint{kl}{ij} \notag \\
  &- \Perm{ab|ij}\tamp{a\bar{j}}{k\bar{c}}\xint{k\bar{c}}{i\bar{b}} - \Perm{ab}\tamp{a}{k}\xxint{k}{ij\bar{b}} + \Perm{ij}\xxint{ab\bar{i}}{c}\tamp{c}{j} \notag \\
  \bxint{pphh}{1} &\longleftarrow \bvint{pphh}{1} + \frac{1}{2}\bxxint{pppp}{1}\cdot\btamp{}{1} + \frac{1}{2}\btamp{}{1}\cdot\bxint{hhhh}{1} \notag \\
  \bxint{pphh}{3_{1(2)}} &\longleftarrow \pm\bxint{pp}{3}\cdot\btamp{}{3_{1}} \mp\btamp{}{3}\bxxint{hphh}{3_{1}} \notag \\
  \bxint{pphh}{3_{3(4)}} &\longleftarrow \mp\btamp{}{3_{3}}\cdot\bxint{hh}{3} \pm\bxxint{pphp}{3_{4}}\btamp{}{3} \notag \\
  \bxint{pphh}{2_{1(2)}} &\longleftarrow -\btamp{}{2_{1}}\cdot\bxint{hphp}{2_{1}} \notag \\
  \bxint{pphh}{2_{3(4)}} &\longleftarrow \btamp{}{2_{1}}\cdot\bxint{hphp}{2_{1}}
\end{align}


\section{Matrix Form of $\left(\EHamN\Rop^{A+1}_{\mu}\right)_{\mathrm{c}} = \omega_{\mu}\Rop^{A+1}_{\mu}$}

\begin{align}
  \omega_{k}\rop{a}{} &= \xint{a}{c}\rop{c}{} + \rop{a}{k\bar{c}}\xint{k\bar{c}}{} - \frac{1}{2}\xint{a}{cd\bar{k}}\rop{cd\bar{k}}{} \notag \\
  \omega_{k}\brop{}{} &\longleftarrow \bxint{pp}{3}\cdot\brop{}{} + \brop{}{2_{1}}\cdot\bxint{hp}{2} - \frac{1}{2}\bxint{hppp}{3_{2}}\cdot\brop{}{3}
\end{align}
\begin{align}
  \omega_{k}\rop{ab}{i} &= -\xint{ab\bar{i}}{c}\rop{c}{} + \Perm{ab}\xint{b}{c}\rop{c}{i\bar{a}} - \rop{ab}{k}\xint{k}{i} + \frac{1}{2}\xint{ab}{cd}\rop{cd}{i} - \Perm{ab}\rop{b}{k\bar{c}}\xint{k\bar{c}}{i\bar{a}}  - \frac{1}{2}\amp{ab\bar{i}}{k}\vint{k}{cd\bar{l}}\rop{cd\bar{l}}{} \notag \\
  \omega_{k}\brop{}{3} &\longleftarrow -\bxint{pphp}{3_{4}}\cdot\brop{}{} - \frac{1}{2}\btamp{}{3_{3}}\cdot\bvint{hhpp}{3_{1}}\cdot\brop{}{3} \notag \\
  \omega_{k}\brop{}{2_{1(2)}} &\longleftarrow \mp\bxint{pp}{3}\cdot\brop{}{2_{2}} \pm\brop{}{2_{2}}\cdot\bxint{hphp}{2_{1}} \notag \\
  \omega_{k}\brop{}{1} &\longleftarrow -\brop{}{1}\cdot\bxint{hh}{3} + \frac{1}{2}\bxint{pppp}{1}\cdot\brop{}{1}
\end{align}


\section{Matrix Form of $\Lop^{A+1}_{\mu}\EHamN = E_{\mu}\Lop^{A+1}_{\mu}$}
One main difference for the left eigenproblem is that the disconnected term is computed as an outer product rather than with matrix-matrix multiplication.

\begin{align}
  E_{k}\lop{}{a} &= \lop{}{c}\xint{c}{a} - \frac{1}{2}\lop{}{cd\bar{k}}\xint{cd\bar{k}}{a} \notag \\
  E_{k}\blop{}{} &\longleftarrow \blop{}{}\cdot\bxint{pp}{3} - \frac{1}{2}\blop{}{3}\cdot\bxint{pphp}{3_{4}}
\end{align} 
\begin{align}
  E_{k}\lop{i}{ab} &= \Perm{ab}\lop{}{a}\xint{i\bar{b}}{} - \lop{}{c}\xint{c}{ab\bar{i}} - \xint{i}{k}\lop{k}{ab} + \Perm{ab}\lop{i\bar{a}}{c}\xint{c}{b} + \frac{1}{2}\lop{i}{cd}\xint{cd}{ab} \notag \\
  &- \Perm{ab}\xint{i\bar{a}}{k\bar{c}}\lop{k\bar{c}}{b} - \frac{1}{2}\lop{}{cd\bar{l}}\tamp{cd\bar{l}}{k}\vint{k}{ab\bar{i}} \notag \\
  E_{k}\lop{}{3} &\longleftarrow -\blop{}{}\cdot\bxint{hppp}{3_{2}} - \frac{1}{2}\blop{}{3}\cdot\btamp{}{3_{3}}\cdot\bvint{hhpp}{3_{1}} \notag \\
  E_{k}\blop{}{2_{1(2)}} &\longleftarrow \pm\blop{}{}\otimes\bxint{hp}{2} \mp\blop{}{2_{2}}\cdot\bxint{pp}{3} \pm\bxint{hphp}{2_{1}}\cdot\blop{}{2_{2}} \notag \\
  E_{k}\blop{}{1} &\longleftarrow -\bxint{hh}{3}\cdot\blop{}{1} + \frac{1}{2}\blop{}{1}\cdot\bxint{pppp}{1}
\end{align}


\section{Matrix Form of $\left(\EHamN\Rop^{A-1}_{\mu}\right)_{\mathrm{c}} = \omega_{\mu}\Rop^{A-1}_{\mu}$}

\begin{align}
  \omega_{k}\rop{}{i} &= -\rop{}{k}\xint{k}{i} + \xint{}{c\bar{k}}\rop{c\bar{k}}{i} - \frac{1}{2}\rop{}{kl\bar{c}}\xint{kl\bar{c}}{i} \notag \\
  \omega_{k}\brop{}{} &\longleftarrow -\brop{}{}\cdot\bxint{hh}{3} + \bxint{hp}{2'}\cdot\brop{}{2_{1}} - \frac{1}{2}\brop{}{3}\cdot\bxint{hhhp}{3_{3}}
\end{align}
\begin{align}
  \omega_{k}\rop{a}{ij} &= -\rop{}{k}\xint{k}{ij\bar{a}} - \Perm{ij}\rop{a\bar{i}}{k}\xint{k}{j} + \xint{a}{c}\rop{c}{ij} + \frac{1}{2}\rop{a}{kl}\xint{kl}{ij} - \Perm{ij}\xint{a\bar{i}}{c\bar{k}}\rop{c\bar{k}}{j} - \frac{1}{2}\rop{}{kl\bar{d}}\vint{kl\bar{d}}{c}\amp{c}{ij\bar{a}} \notag \\
  \omega_{k}\brop{}{3} &\longleftarrow -\brop{}{}\cdot\bxint{hphh}{3_{1}} - \frac{1}{2}\brop{}{3}\cdot\bvint{hhpp}{3_{3}}\cdot\btamp{}{3_{1}} \notag \\
  \omega_{k}\brop{}{2_{1(2)}} &\longleftarrow \pm\brop{}{2_{2}}\cdot\bxint{hh}{3} \pm\bxint{hphp}{2_{2}}\cdot\brop{}{2_{2}} \notag \\
  \omega_{k}\brop{}{1} &\longleftarrow \bxint{pp}{3}\cdot\brop{}{1} + \frac{1}{2}\brop{}{1}\cdot\bxint{hhhh}{1}
\end{align}


\section{Matrix Form of $\Lop^{A-1}_{\mu}\EHamN = E_{\mu}\Lop^{A-1}_{\mu}$}
Again, the disconnected term is computed as an outer product rather than with matrix-matrix multiplication.

\begin{align}
  E_{k}\lop{i}{} &= -\xint{i}{k}\lop{k}{} - \frac{1}{2}\xint{i}{kl\bar{c}}\lop{kl\bar{c}}{} \notag \\
  E_{k}\blop{}{} &\longleftarrow -\bxint{HH}{3}\cdot\blop{}{} - \frac{1}{2}\bxint{hphh}{3_{1}}\cdot\blop{}{3}
\end{align}
\begin{align}
  E_{k}\lop{ij}{a} &= \Perm{ij}\lop{i}{}\xint{}{a\bar{j}} - \xint{ij\bar{a}}{k}\lop{k}{} + \lop{ij}{c}\xint{c}{a} - \Perm{ij}\xint{j}{k}\lop{k}{a\bar{i}} + \frac{1}{2}\xint{ij}{kl}\lop{kl}{a} \notag \\
  &- \Perm{ij}\lop{j}{c\bar{k}}\xint{c\bar{k}}{a\bar{i}} - \frac{1}{2}\vint{ij\bar{a}}{c}\amp{c}{kl\bar{d}}\lop{kl\bar{d}}{} \notag \\
  E_{k}\lop{}{3} &\longleftarrow -\bxint{hhhp}{3_{3}}\cdot\blop{}{} - \frac{1}{2}\bvint{hhpp}{3_{3}}\cdot\btamp{}{3_{1}}\cdot\blop{}{3} \notag \\
  E_{k}\blop{}{2_{1(2)}} &\longleftarrow \pm\blop{}{}\otimes\bxint{hp}{2'} \pm\bxint{hh}{3}\cdot\blop{}{2_{2}} \pm\blop{}{2_{2}}\cdot\bxint{hphp}{2_{2}} \notag \\
  E_{k}\blop{}{1} &\longleftarrow \blop{}{1}\cdot\bxint{pp}{3} + \frac{1}{2}\bxint{hhhh}{1}\cdot\blop{}{1}
\end{align}



\chapter{Angular Momentum Coupling} \label{chapter:angular_momentum}

Before deriving useful equations for J-scheme angular momentum coupling, it's necessary to list some shorthand notations, definitions, and useful relationships:
\begin{gather}
  \hat{p} \equiv \sqrt{2 j_{p} + 1} \\
  \sum_{\{ m \}} \equiv \text{sum over all}\ m
\end{gather}

Clebsch-Gordan coefficients:
\begin{gather}
  \braket{pq}{J} \equiv \braket{j_{p}m_{p} ; j_{q}m_{q}}{JM} \\
  \braket{p\bar{q}}{J} \equiv \braket{j_{p},m_{p} ; j_{q},-m_{q}}{JM}(-1)^{( q - m_{q} )} \\
  \sum_{JM} \braket{pq}{J}\braket{p'q'}{J} = \delta_{m_{p}m_{p'}}\delta_{m_{q}m_{q'}} \\
  \label{eq:cgc_identity}
  \sum_{m_{p}m_{q}} \braket{pq}{J}\braket{pq}{J'} = \delta_{JJ'}\delta_{MM'} \\
\end{gather}

Clebsch-Gordan coefficient symmetries:
\begin{align}
  \label{eq:cgc1}
  \braket{j_{p}m_{p} ; j_{q}m_{q}}{JM} &= (-1)^{j_{p}+j_{q}-J}\braket{j_{p}m_{p} ; j_{q}m_{q}}{JM} \\
  \label{eq:cgc2}
  &= (-1)^{j_{p}+j_{q}-J}\braket{j_{q}m_{q} ; j_{p}m_{p}}{JM} \\
  &= (-1)^{j_{p}-m_{p}}\frac{\hat{J}}{\hat{q}}\braket{j_{p}m_{p} ; J-M}{j_{q}-m_{q}} \\
  &= (-1)^{j_{q}+m_{q}}\frac{\hat{J}}{\hat{p}}\braket{J-M ; j_{q}m_{q}}{j_{p}-m_{p}} \\
  \label{eq:cgc5}
  &= (-1)^{j_{p}-m_{p}}\frac{\hat{J}}{\hat{q}}\braket{JM ; j_{p}-m_{p}}{j_{q}m_{q}} \\
  &= (-1)^{j_{q}+m_{q}}\frac{\hat{J}}{\hat{p}}\braket{j_{q}-m_{q} ; JM}{j_{p}m_{p}}
\end{align}
  
Six-J symbols:
\begin{gather}
  \sixj{p}{q}{r}{s}{t}{u} \equiv \sixj{j_{p}}{j_{q}}{j_{r}}{j_{s}}{j_{t}}{j_{u}} \\
  \sum_{j_{3}} \hat{j}^{2}_{3} \sixj{j_{1}}{j_{2}}{j_{3}}{j_{4}}{j_{5}}{j_{6}} \sixj{j_{1}}{j_{2}}{j_{3}}{j_{4}}{j_{5}}{j'_{6}} = \frac{\delta_{j_{6}j'_{6}}}{\hat{j}^{2}_{3}} \\
  \label{eq:pandya1}
  \sum_{M'} \braket{p\bar{s}}{J'}\braket{r\bar{q}}{J'} = \hat{J}'^{2}\sum_{JM}\sixj{p}{q}{J}{r}{s}{J'}\braket{pq}{J}\braket{rs}{J} \\
  \label{eq:pandya2}
  \sum_{M} \braket{pq}{J}\braket{rs}{J} = \hat{J}^{2}\sum_{J'M'}\sixj{p}{q}{J}{r}{s}{J'}\braket{p\bar{s}}{J'}\braket{r\bar{q}}{J'}
\end{gather}

Two-body, scalar J-scheme matrix elements ($\Top, \Ham, \EHam$), in terms of M-scheme matrix elements:
\begin{align}
  X^{pq^{J}}_{rs^{J}} &= \sum_{\{ m \}} X^{p_{m_{p}} q_{m_{q}}}_{r_{m_{r}} s_{m_{s}}} \braket{pq}{J} \braket{rs}{J} \\
  \label{eq:jscheme2}
  X^{p\bar{s}^{J'}}_{r\bar{q}^{J'}} &= \sum_{\{ m \}} X^{p_{m_{p}} q_{m_{q}}}_{r_{m_{r}} s_{m_{s}}} \braket{p\bar{s}}{J'} \braket{r\bar{q}}{J'} \\
  \label{eq:jscheme3}
  X^{p}_{rs^{J}\bar{q}} &= \sum_{\{ m \}} X^{p_{m_{p}} q_{m_{q}}}_{r_{m_{r}} s_{m_{s}}} \braket{rs}{J} \braket{J\bar{q}}{p} \\
  X^{pq^{J}\bar{s}}_{r} &= \sum_{\{ m \}} X^{p_{m_{p}} q_{m_{q}}}_{r_{m_{r}} s_{m_{s}}} \braket{pq}{J} \braket{J\bar{s}}{r}
\end{align}}
Two-body M-scheme matrix elements in terms of J-scheme, scalar matrix elements:
\begin{align}
  X^{p_{m_{p}} q_{m_{q}}}_{r_{m_{r}} s_{m_{s}}} &= \sum_{JM} X^{pq^{J}}_{rs^{J}} \braket{pq}{J} \braket{rs}{J} \\
  \label{eq:mscheme2}
  &= \sum_{J'M'} X^{p\bar{s}^{J'}}_{r\bar{q}^{J'}} \braket{p\bar{s}}{J'} \braket{r\bar{q}}{J'} \\
  \label{eq:mscheme3}
  &= \sum_{JM} X^{p}_{rs^{J}\bar{q}} \braket{rs}{J} \braket{J\bar{q}}{p} \\
  &= \sum_{JM} X^{pq^{J}\bar{s}}_{r} \braket{pq}{J} \braket{J\bar{s}}{r}
\end{align}

To find the relationship between the scalar matrix elements of $\Top$, $\Ham$, and $\EHam$ with different couplings, the M-scheme expressions are written in terms of their different couplings, then the Clebsch-Gordon coefficients are reorganized using Eqs.\ \eqref{eq:cgc1}--\eqref{eq:pandya2} so that they have the same form.  A few examples of this recoupling are shown below with the relevant equation used at each step. The shorthand $X \equiv X^{p_{m_{p}} q_{m_{q}}}_{r_{m_{r}} s_{m_{s}}}$ is used for clarity.  The relationship between $X^{pq^{J}}_{rs^{J}}$ and $X^{p\bar{s}^{J'}}_{r\bar{q}^{J'}}$ is,
\begin{align} \label{eq:structure_trans1}
  X^{pq^{J}}_{rs^{J}} &= \sum_{\mathclap{ M\{ m \} }} X \braket{pq}{J}\braket{rs}{J} = \sum_{\mathclap{ J'M'\{ m \} }} X \hat{J}^{2}\sixj{p}{q}{J}{r}{s}{J'}\braket{p\bar{s}}{J'}\braket{r\bar{q}}{J'}\ \hspace{0.5cm} \eqref{eq:pandya2}  \notag \\
  &= \sum_{\mathclap{ J }} X^{p\bar{s}^{J'}}_{r\bar{q}^{J'}} \hat{J}^{2}\sixj{p}{q}{J}{r}{s}{J'}\ \hspace{0.5cm} \eqref{eq:jscheme2}
\end{align}
As another example, the relationship between $X^{pq^{J}}_{rs^{J}}$ and $X^{p}_{rs^{J}\bar{q}}$ is,
\begin{align} \label{eq:structure_trans2}
  X^{pq^{J}}_{rs^{J}} &= \sum_{\mathclap{ \{ m \} }} X \braket{pq}{J}\braket{rs}{J} = \sum_{\mathclap{ \{ m \} }} X \braket{qp}{J}\braket{rs}{J}(-1)^{j_{p}+j_{q}-J}\ \hspace{0.5cm} \eqref{eq:cgc2}  \notag \\
  &= \sum_{\mathclap{ \{ m \} }} X \frac{\hat{J}}{\hat{p}}\braket{J\bar{q}}{p}\braket{rs}{J}(-1)^{j_{p}+j_{q}-J}\ \hspace{0.5cm} \eqref{eq:cgc5}  \notag \\
  &= X^{p}_{rs^{J}\bar{q}} \frac{\hat{J}}{\hat{p}}(-1)^{j_{p}+j_{q}-J}\ \hspace{0.5cm} \eqref{eq:jscheme2}
\end{align}

When sums are formulated in the terms of channel-partitioned matrices like those in sections \ref{section:maxtrix_intermediates} and \ref{section:channel_matrices}, the factors related to a structure's angluar momentum coupling are automatically summed with the identity Eq.\ \eqref{eq:cgc_identity}.  To demonstrate this, an example is shown here.  First, from the CCSD equations the sum in Eq.\ \eqref{eq:amp_matrices_3} is rewritten in terms of the J-scheme structures.  The indices represent single-particle states in the M-scheme expression but represent degenerate shells in J-scheme,
\begin{gather}
  \frac{1}{2}\sum\limits_{\mathclap{klcd}}\vint{kl}{cd}\tamp{db}{ij}\tamp{ca}{kl}\ \longrightarrow\ \frac{1}{2}\sum\limits_{\mathclap{\substack{klcd \\ J_{1}J_{2}J_{3} \\ \{ m \} }}} \tamp{a}{kl^{J_{1}}\bar{c}}\braket{kl}{J_{1}}\braket{J_{1}\bar{c}}{a} \vint{kl^{J_{2}}\bar{c}}{d}\braket{kl}{J_{2}}\braket{J_{2}\bar{c}}{d} \tamp{d}{ij^{J_{3}}\bar{b}}\braket{ij}{J_{3}}\braket{J_{3}\bar{b}}{d} \notag \\
  =\frac{1}{2}\sum\limits_{\mathclap{\substack{klcd \\ J_{1}J_{2}J_{3} \\ \{ m \} }}} \tamp{a}{kl^{J_{1}}\bar{c}} \vint{kl^{J_{2}}\bar{c}}{d} \tamp{d}{ij^{J_{3}}\bar{b}} \left[ \sum\limits_{m_{k}m_{l}} \braket{kl}{J_{1}}\braket{kl}{J_{2}} \right] \braket{J_{1}\bar{c}}{a} \braket{J_{2}\bar{c}}{d} \braket{ij}{J_{3}} \braket{J_{3}\bar{b}}{d} \notag \\
  =\frac{1}{2}\sum\limits_{\mathclap{\substack{klcd \\ J_{1}J_{2}J_{3} \\ \{ m \} }}} \tamp{a}{kl^{J_{1}}\bar{c}} \vint{kl^{J_{2}}\bar{c}}{d} \tamp{d}{ij^{J_{3}}\bar{b}} \left[ \delta_{J_{1}J_{2}}\delta_{M_{1}M_{2}} \right] \braket{J_{1}\bar{c}}{a} \braket{J_{2}\bar{c}}{d} \braket{ij}{J_{3}} \braket{J_{3}\bar{b}}{d} \notag \\
  =\frac{1}{2}\sum\limits_{\mathclap{\substack{klcd \\ J_{1}J_{3} \\ \{ m \} }}} \tamp{a}{kl^{J_{1}}\bar{c}} \vint{kl^{J_{1}}\bar{c}}{d} \tamp{d}{ij^{J_{3}}\bar{b}} \left[ \sum\limits_{M_{1}m_{c}} \braket{J_{1}\bar{c}}{a} \braket{J_{1}\bar{c}}{d} \right] \braket{ij}{J_{3}} \braket{J_{3}\bar{b}}{d} \notag \\
  =\frac{1}{2}\sum\limits_{\mathclap{\substack{klcd \\ J_{1}J_{3} \\ \{ m \} }}} \tamp{a}{kl^{J_{1}}\bar{c}} \vint{kl^{J_{1}}\bar{c}}{d} \tamp{d}{ij^{J_{3}}\bar{b}} \left[ \delta_{j_{a}j_{d}}\delta_{m_{a}m_{d}} \right] \braket{ij}{J_{3}} \braket{J_{3}\bar{b}}{d} \notag \\
  =\frac{1}{2}\sum\limits_{\mathclap{\substack{klcd \\ J_{1}J_{3} \\ \{ m \} }}} \tamp{a}{kl^{J_{1}}\bar{c}} \vint{kl^{J_{1}}\bar{c}}{d} \tamp{d}{ij^{J_{3}}\bar{b}} \braket{ij}{J_{3}} \braket{J_{3}\bar{b}}{a}
\end{gather}
This final form has the same structure as Eq.\ \eqref{eq:mscheme3} so that the sum can be collected into the following structure with no angular-momenutm coupling coefficients,
\begin{equation}
  \tamp{a}{ij^{J}\bar{b}}\ \leftarrow \frac{1}{2}\sum\limits_{\mathclap{\substack{klcd \\ J_{1}J_{3} }}} \tamp{a}{kl^{J_{1}}\bar{c}}\vint{kl^{J_{1}}\bar{c}}{d}\tamp{d}{ij^{J_{3}}\bar{b}}.
\end{equation}
Amplitudes of different coupling structures are then gathered using relationships like those in Eq.\ \eqref{eq:structure_trans1} and \eqref{eq:structure_trans2}.


\end{document}
