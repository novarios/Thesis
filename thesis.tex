\documentclass[dissertation]{msudissertation}

\usepackage{amsmath}
\usepackage{amssymb}
\usepackage{mathtools}
\usepackage{subfiles}  % provides ability to use subfiles for sections
\usepackage{graphicx}
\usepackage{subcaption}
\graphicspath{{figures/}{../figures/}}
\usepackage{framed}
\usepackage{dashbox}
\usepackage{simplewick}

% optional packages
\usepackage{hyperref}  % provides hyperlinks (\url) and PDF bookmarks
\usepackage{pdflscape} % provides \begin{landscape} ... \end{landscape}
\usepackage{titling}   % provides \thetitle, \theauthor, \thedate

\def\dheight{2.5cm}
\def\dscale{1.15}
\def\sdheight{1.7cm}

\newcommand*\xbar[1]{
  \hbox{
    \vbox{
      \hrule height 0.5pt
      \kern0.35ex
      \hbox{
        \kern-0.55em
        \ensuremath{#1}
        \kern-0.41em
      }
    }
  }
}

\newcommand{\D}{\operatorname{d\!}}
\newcommand{\E}{\operatorname{e}}
\newcommand{\bra}[1]{\langle #1 |\hspace{1pt}}
\newcommand{\braket}[2]{\langle #1 | #2\rangle}
\newcommand{\ket}[1]{\hspace{1pt}\lvert #1 \rangle}
\newcommand{\rbra}[1]{\left\langle #1 \right\|}
\newcommand{\rket}[1]{\left\| #1 \right\rangle}
\newcommand{\element}[3]{\bra{#1}#2\ket{#3}}
\newcommand{\relement}[3]{\rbra{#1}#2\rket{#3}}
\newcommand{\normord}[1]{\left\{#1\right\}}
\newcommand{\antisymm}{\mathop{\scalerel*{\mathcal{A}}{\textstyle\sum}}\displaylimits}

\newcommand{\Ecorr}{\Delta E}
\newcommand{\Ham}{\hat{H}\hspace{-1pt}}
\newcommand{\HamB}[1]{{}^{\left(#1\right)}\hspace{-2pt}\hat{H}\hspace{-2pt}}
\newcommand{\EHam}{\xbar{H}\hspace{-2pt}}
\newcommand{\HamN}{\hat{H}_{\mathrm{N}}}
\newcommand{\EHamN}{\xbar{H}_{\hspace{-0.75ex}\mathrm{N}}}
\newcommand{\EOp}{\xbar{O}\hspace{-2pt}}
\newcommand{\Top}{\hat{T}\hspace{-2pt}}
\newcommand{\Vop}{\hat{V}\hspace{-2pt}}
\newcommand{\Wop}{\hat{W}\hspace{-2pt}}
\newcommand{\Rop}{\hat{R}\hspace{-0.1pt}}
\newcommand{\Lop}{\hat{L}\hspace{-0.1pt}}
\newcommand{\Lamdaop}{\hat{\Lambda}}
\newcommand{\Edenom}[2]{\varepsilon^{#1}_{#2}}
\newcommand{\Vt}{\tilde{V}\hspace{-2pt}}
\newcommand{\Res}{\hat{R}_{0}\hspace{-2pt}}

\newcommand{\Ref}{\Phi_{0}}
\newcommand{\statebra}[2]{\bra{\Phi^{#1}_{#2}}}
\newcommand{\stateket}[2]{\ket{\Phi^{#1}_{#2}}}
\newcommand{\vacket}{\ket{0}}
\newcommand{\refket}{\ket{\Ref}}
\newcommand{\corrket}{\ket{\Psi}}
\newcommand{\vacbra}{\bra{0}}
\newcommand{\refbra}{\bra{\Ref}}
\newcommand{\corrbra}{\bra{\Psi}}

\newcommand{\ph}[2]{#1p\hspace{2pt}\text{-}#2h}
\newcommand{\Perm}[1]{\mathop{\hat{P}\hspace{-2pt}\left( #1\right) }}
\newcommand{\amp}[2]{\text{t}^{#1}_{#2}}
\newcommand{\tamp}[2]{t^{#1}_{#2}}
\newcommand{\lamp}[2]{\lambda^{#1}_{#2}}
\newcommand{\fint}[2]{f^{#1}_{#2}}
\newcommand{\vint}[2]{\text{V}^{#1}_{#2}}
\newcommand{\wint}[2]{\text{W}^{#1}_{#2}}
\newcommand{\KEint}[2]{T^{#1}_{#2}}
\newcommand{\Hint}[3]{{}^{\left(#1\right)}\hspace{-2pt}H^{#2}_{#3}}
\newcommand{\xint}[2]{X^{#1}_{\hspace{0.3pt} #2}}
\newcommand{\xxint}[2]{X'^{#1}_{\hspace{2.3pt} #2}}
\newcommand{\xxxint}[2]{X''^{#1}_{\hspace{4.3pt} #2}}
\newcommand{\xxxxint}[2]{X'''^{#1}_{\hspace{6.3pt} #2}}
\newcommand{\opint}[3]{{}^{#1}O^{\hspace{0.5pt} #2}_{#3}}
\newcommand{\oopint}[3]{{}^{#1}\hspace{-1pt}\xbar{O}^{\hspace{-2.5pt} \raisebox{-1.5pt}{\footnotesize $#2$}}_{\hspace{-3pt} #3}}
\newcommand{\rop}[2]{r^{#1}_{#2}}
\newcommand{\lop}[2]{l^{#1}_{#2}}

\newcommand{\bamp}[2]{\text{\textbf{t}}^{#1}_{#2}}
\newcommand{\btamp}[2]{\mathbf{t}^{#1}_{#2}}
\newcommand{\blamp}[2]{\mathbf{\lambda}^{#1}_{#2}}
\newcommand{\bfint}[2]{\mathbf{f}^{#1}_{#2}}
\newcommand{\bvint}[2]{\text{\textbf{V}}^{#1}_{#2}}
\newcommand{\bwint}[2]{\text{\textbf{W}}^{#1}_{#2}}
\newcommand{\bKEint}[2]{\mathbf{T}^{#1}_{#2}}
\newcommand{\bHint}[3]{{}^{\left(#1\right)}\hspace{-2pt}\mathbf{H}^{#2}_{#3}}
\newcommand{\bxint}[2]{\mathbf{X}^{#1}_{\hspace{0.3pt} #2}}
\newcommand{\bxxint}[2]{\mathbf{X'}^{#1}_{\hspace{2.3pt} #2}}
\newcommand{\bxxxint}[2]{\mathbf{X''}^{#1}_{\hspace{4.3pt} #2}}
\newcommand{\bxxxxint}[2]{\mathbf{X'''}^{#1}_{\hspace{6.3pt} #2}}
\newcommand{\bopint}[3]{{}^{#1}\mathbf{O}^{\hspace{0.5pt} #2}_{#3}}
\newcommand{\boopint}[3]{{}^{#1}\hspace{-1pt}\xbar{\mathbf{O}}^{\hspace{-2.5pt} \raisebox{-1.5pt}{\footnotesize $#2$}}_{\hspace{-3pt} #3}}
\newcommand{\brop}[2]{\mathbf{r}^{#1}_{#2}}
\newcommand{\blop}[2]{\mathbf{l}^{#1}_{#2}}

\newcommand{\co}[1]{\hat{a}_{#1}^{\dagger}}
\newcommand{\ao}[1]{\hat{a}_{#1}^{}}
\newcommand{\cco}[1]{\hat{#1}^{\dagger}}
\newcommand{\aao}[1]{\hat{#1}^{}}

\newcommand{\diagram}[1]{\hspace{1.15mm}\vcenter{\hbox{\includegraphics[height=\dheight]{diagrams/#1.pdf}}}\hspace{1.15mm}}
\newcommand{\fdiagram}[1]{\hspace{-0.75mm}\vcenter{\hbox{\includegraphics[height=\dheight]{diagrams/#1.pdf}}}\hspace{1.15mm}}
\newcommand{\ddiagram}[1]{\hspace{1.15mm}\vcenter{\hbox{\dbox{\includegraphics[height=\dheight]{diagrams/#1.pdf}}}}\hspace{1.15mm}}
\newcommand{\sdiagram}[1]{\hspace{0.8mm}\vcenter{\hbox{\includegraphics[height=\sdheight]{diagrams/#1.pdf}}}\hspace{0.8mm}}
\newcommand{\sddiagram}[1]{\hspace{0.8mm}\vcenter{\hbox{\dbox{\includegraphics[height=\sdheight]{diagrams/#1.pdf}}}}\hspace{0.8mm}}

\newcommand{\dboxed}[1]{\dbox{\ensuremath{#1}}}

\newcommand{\sixj}[6]{\begin{Bmatrix} #1 & #2 & #3 \\ #4 & #5 & #6 \end{Bmatrix}}

\author{Samuel John Novario}
\title{Effective Nuclear Operators with the Coupled-Cluster Method}
\date{2017}
\def\thedegreeprogram{Physics---Doctor of Philosophy}

\begin{document}

\frontmatter
\maketitle

\chapter{Public Abstract}

Your public abstract goes here.  This is a preview of the dissertation/thesis template.  General instructions for the template can be found in Chapter \ref{chapter:instructions}.

---

Lorem ipsum dolor sit amet, consectetur adipiscing elit. Vivamus tristique pretium ipsum nec bibendum. Vestibulum eleifend viverra dui, non molestie libero tincidunt a. Duis commodo odio eget rhoncus cursus. Quisque mattis scelerisque purus in facilisis. Ut consectetur luctus venenatis. Phasellus pulvinar congue tellus, eu tempor augue congue at. In hac habitasse platea dictumst. In accumsan tristique neque quis convallis. Duis quis porttitor orci. Vestibulum ante ipsum primis in faucibus orci luctus et ultrices posuere cubilia Curae; Sed sit amet elit sit amet elit scelerisque mattis. Nam a cursus sapien.

\chapter{Abstract}

Your abstract goes here.  This is a preview of the dissertation/thesis template.  General instructions for the template can be found in Chapter \ref{chapter:instructions}.

---

Lorem ipsum dolor sit amet, consectetur adipiscing elit. Vivamus tristique pretium ipsum nec bibendum. Vestibulum eleifend viverra dui, non molestie libero tincidunt a. Duis commodo odio eget rhoncus cursus. Quisque mattis scelerisque purus in facilisis. Ut consectetur luctus venenatis. Phasellus pulvinar congue tellus, eu tempor augue congue at. In hac habitasse platea dictumst. In accumsan tristique neque quis convallis. Duis quis porttitor orci. Vestibulum ante ipsum primis in faucibus orci luctus et ultrices posuere cubilia Curae; Sed sit amet elit sit amet elit scelerisque mattis. Nam a cursus sapien.

\chapter{Copyright}

(Only needed if you intend \\
to register for a copyright.)
\\
---
\\
Copyright by \\
\MakeUppercase{\theauthor} \\
\thedate

\chapter{Dedication}

Your dedication goes here.  It is optional.

---

Lorem ipsum dolor sit amet, consectetur adipiscing elit. Donec mollis mauris vitae massa aliquet, at accumsan lorem lobortis.

\chapter{Acknowledgments}

Your acknowledgments goes here.  It is optional.  You can also use the variant spelling ``Acknowledgements'' (note the extra ``e'').

---

Lorem ipsum dolor sit amet, consectetur adipiscing elit. Donec vel nisl aliquam, tincidunt lacus et, sollicitudin lectus. Mauris quis sagittis risus. Lorem ipsum dolor sit amet, consectetur adipiscing elit. Vestibulum aliquet odio leo, eget laoreet leo dictum luctus. Aliquam at nisi eu turpis posuere rhoncus id ac neque. Nunc iaculis turpis id rhoncus rutrum.

In pharetra neque luctus, vestibulum diam quis, vehicula turpis. Proin magna magna, feugiat sed luctus consectetur, vehicula id erat. Phasellus non nisi ac ipsum vehicula aliquet. Proin pulvinar sit amet metus id finibus. Mauris et justo et mi sollicitudin commodo. Cras tempus interdum lectus ac iaculis. Proin purus purus, euismod non magna aliquet, gravida volutpat velit. Praesent eu erat purus. Cras tempus eu dolor vitae malesuada.

\chapter{Preface}

Your preface goes here.  It is optional.  This is a preview of the dissertation/thesis template.  General instructions for the template can be found in Chapter \ref{chapter:instructions}.

---

Lorem ipsum dolor sit amet, consectetur adipiscing elit. In id pellentesque lacus. Praesent scelerisque eros sit amet felis faucibus pretium. Pellentesque cursus maximus consectetur. Suspendisse at congue eros. Cras tincidunt tellus lorem, ut viverra ligula sagittis eget. Praesent egestas viverra leo a rutrum. Vivamus finibus magna eu sapien tristique, eget congue mi facilisis. Integer porttitor dignissim dolor ut porta. Nulla congue hendrerit nulla, et volutpat elit scelerisque vitae.

Donec convallis, nunc eget efficitur lacinia, magna ipsum dignissim dolor, vitae mollis neque sem in ipsum. In quam mi, laoreet ac nibh sed, imperdiet tincidunt tellus. Sed dictum ante ac facilisis ornare. Curabitur sit amet purus quis nulla dictum blandit in ut purus. Praesent eget ligula in nisl interdum facilisis vitae vitae magna. Ut quis ornare augue. Cras aliquet ac ex iaculis rutrum. Vestibulum volutpat fermentum orci, sed laoreet neque auctor eu. Fusce ac lorem congue, blandit eros a, tincidunt leo. Cras luctus ultricies mollis. Phasellus laoreet nulla sit amet ipsum sollicitudin facilisis. Donec sit amet est volutpat, accumsan nisi sed, fringilla mi.

\tableofcontents
\listoftables
\listoffigures

\chapter{Key to Symbols and Abbreviations}
\subfile{SymbolsAbbreviations}

\mainmatter

\chapter{Introduction} \label{chapter:introduction}
\subfile{introduction}

\chapter{Many-Body Quantum Mechanics} \label{chapter:manybody}
\subfile{ManyBody}

\chapter{Coupled-Cluster Theory} \label{chapter:cc}
\subfile{CC}

\chapter{Equation-of-Motion Method} \label{chapter:eom}
\subfile{EOM}

\chapter{Effective Operators} \label{chapter:effectiveoperators}
\subfile{EffectiveOperator}

\section{Beta Decay} \label{chapter:betadecay}
\subfile{BetaDecay}

\chapter{Conclusions and Perspectives} \label{chapter:conclusions}

\chapter{Instructions} \label{chapter:instructions}

If you want to use this class, it's probably a good idea to use the source code of this example document is as a starting point.

\section{Preamble}

The document class may be declared using \verb|\documentclass[<type>]{msudissertation}|, where \texttt{<type>} is either \texttt{dissertation} (default) or \texttt{thesis}.  The class is based on the \texttt{book} class and thus inherits all its structural conventions.  The WikiBooks has more information about this: \url{https://en.wikibooks.org/wiki/LaTeX/Document_Structure} .

Afterward, you can load your packages.  Among those, you may find the following packages useful:
\begin{itemize}
\item \verb|\usepackage{hyperref}|: provides hyperlinks (\verb|\url|) and PDF bookmarks
\item \verb|\usepackage{pdflscape}|: provides \verb|\begin{landscape} |\ldots\verb| \end{landscape}|
\item \verb|\usepackage{titling}|: provides \verb|\thetitle|, \verb|\theauthor|, and \verb|\thedate|
\end{itemize}
Refer to their official documentation on \href{https://www.ctan.org}{CTAN} for more details.

To set the title, author, degree program, and date, include the following commands in your preamble:
\begin{itemize}
\item \verb|\title{<title>}|
\item \verb|\author{<name>}|
\item \verb|\def\thedegreeprogram{<subject>---<degree>}|
\item \verb|\date{<year>}| (optional)  Per university guidelines, the date must be contain only a 4-digit year.  If omitted, it defaults to the current year, which could be undesirable if you want the reproduce the document years later.
\end{itemize}

The rest of the document is divided into three major parts, preceded the special markers \verb|\frontmatter|, \verb|\mainmatter|, and \verb|\appendix| respectively.

\section{Front matter}

In the front matter, certain chapter names have been endowed with special meanings, e.g.\ \verb|\chapter{Abstract}| or \verb|\chapter{Copyright}|.  This is what allows them to have unique formatting.  They must be spelled and capitalized exactly as written in the source code of this example, unless otherwise specified.  Given that the \verb|\chapter| command has been imbued with some rather fragile (read: hacky) logic, try not to sneeze on them too hard.

The front matter in this example is a lot more packed than your typical dissertation or thesis, because a lot of the chapters are optional and have been filled with placeholder text.  They can be safely deleted if neither you nor the university guidelines require them.

\section{Main matter}

The main matter is the most uninteresting part of this template, because it's almost the same as your vanilla \texttt{book} class.  Just write \verb|\chapter{<chapter>}| and \verb|\section{<section>}| like you normally do.

\chapter{Appendix} \label{chapter:appendix}

You can have as many appendices as you want, or none at all.  If you do have \emph{at least one}, use the \verb|\appendix| macro to create a cover page with the correct grammatical number depending on how many you have, and adjusts the table of contents according to university guidelines.  After this macro, all uses of \verb|\chapter{<chapter>}| will create an appendix chapter instead of a regular chapter.  Do \emph{not} use \verb|\appendix| if you have no appendices at all.

At the very end, there is the mandatory bibliography.  Here, I'm assuming you want to use the traditional \texttt{natbib}.  If so, start by selecting a \verb|\bibliographystyle{<style>}|.  If you don't like calling it ``Bibliography'', you can pick a more suitable title using the syntax \verb|\renewcommand{\bibname}{<title>}| as long as it conforms to university guidelines.  Afterward, you can write \verb|\bibliography{<name>}| where \texttt{<name>} is the path to the \texttt{.bib} database without the file extension.

And that's it!  The remaining part of this document is full of placeholder text so you can stop reading now.

\chapter{Lorem ipsum}

Lorem ipsum dolor sit amet, consectetur adipiscing elit. Ut et leo non tortor viverra sodales. Ut condimentum odio orci, a varius sapien vehicula quis. Pellentesque a lacus sed sem gravida scelerisque. Nunc rutrum ornare fringilla. Donec ac lorem non leo tincidunt finibus quis in lectus.\cite{exampleref1}

Pellentesque ac consequat leo. Nulla vel aliquet ex. Nulla non faucibus sapien, eu porttitor ligula:

\begin{equation}
  I = \int_{-\infty}^\infty \frac{x}{1 + e^{-x^2 / 2}} \, d x
\end{equation}

Class aptent taciti sociosqu ad litora torquent per conubia nostra, per inceptos himenaeos. In odio metus, maximus sed dictum vel, rhoncus id ante. Phasellus viverra sodales neque ut pulvinar.

Sed neque metus, elementum at risus in, imperdiet iaculis dolor. Vivamus quis libero quis dui finibus imperdiet. Nunc mollis odio eget nibh volutpat interdum. Praesent non felis in sem luctus gravida.

\section{Pellentesque Scelerisque}

\begin{figure}
  \centering
  \rule{6cm}{4cm}
  \caption{Lorem ipsum dolor sit amet, consectetur adipiscing elit. Ut et leo non tortor viverra sodales.}
\end{figure}

Pellentesque scelerisque justo in pellentesque iaculis. Nunc in elementum tortor. Vestibulum tincidunt lacus ac libero gravida, sit amet rhoncus risus iaculis. Sed rhoncus nisl ac felis auctor efficitur. Nulla vel lorem bibendum, laoreet risus vel, euismod odio. Sed a justo augue. Sed sed viverra magna. Etiam vestibulum tellus ut consequat ultricies. Pellentesque mattis odio ac ipsum luctus, ac laoreet ex efficitur. Sed tincidunt, est id viverra ullamcorper, orci dolor accumsan erat, id accumsan tortor est quis ante. Integer pretium elit a ligula tristique imperdiet.

In vitae elit massa. Morbi fermentum arcu quis tristique pretium. Suspendisse potenti. Suspendisse at nisi nec sem dignissim mattis finibus ac sem. Aenean lobortis volutpat turpis, a lacinia ante pellentesque eget. Vivamus fermentum tempus viverra. Sed sollicitudin cursus rutrum. \cite{exampleref1}

\subsection{Fusce Convallis}

\begin{figure}
  \centering
  \rule{5cm}{3cm}
  \caption{Ut condimentum odio orci, a varius sapien vehicula quis.}
\end{figure}

Fusce convallis placerat porta. Nam porta euismod justo, nec efficitur erat tincidunt non. Proin viverra sagittis nibh, ac rhoncus nibh condimentum in. Nunc tristique augue quis ante dapibus, sit amet porta purus molestie. Pellentesque sagittis tincidunt eros, eget fermentum nibh viverra dapibus. Phasellus id dignissim sapien. Mauris eget interdum sapien, in hendrerit ante.

Fusce sed eros sem. Mauris posuere egestas risus et sollicitudin. Suspendisse posuere nec lectus a iaculis. In tincidunt nisl consequat sem feugiat mollis. Vivamus ante libero, lobortis id erat vel, fringilla sodales justo. Mauris vel iaculis lectus, non vehicula ipsum. Praesent mollis convallis nibh in ornare. Fusce tincidunt libero sit amet efficitur convallis.

\section{Curabitur}

Curabitur dignissim orci quis orci blandit blandit.\footnote{Aenean a semper dolor.} Cras cursus tellus quam, sed lacinia nisi scelerisque posuere. Vestibulum sed lacinia orci, id sagittis lorem. Etiam lacinia elementum tortor, ut rutrum tellus laoreet sit amet. Etiam dignissim sagittis sapien in dignissim. Vestibulum ante ipsum primis in faucibus orci luctus et ultrices posuere cubilia Curae; Nullam blandit est vel diam hendrerit posuere. Maecenas nec arcu eu dui semper porttitor.

\begin{landscape}
\begin{table}
  \centering
  \begin{tabular}{l | c | r}
    X & O &   \\ \hline
    O & X &   \\ \hline
    O &   & X \\
  \end{tabular}
  \caption{Nulla suscipit ultricies massa at sagittis.}
\end{table}

Suspendisse aliquet vel orci volutpat imperdiet. Aenean et ex a nisl volutpat ultrices. Duis pellentesque mi vitae maximus sagittis. Quisque dignissim ante pulvinar, porta odio sed, hendrerit ante. Mauris euismod enim ac nibh laoreet, vel eleifend lacus euismod. Fusce bibendum malesuada magna vitae efficitur.
\end{landscape}

\appendix

\chapter{Etiam a Convallis}

Your appendix goes here.

---

Lorem ipsum dolor sit amet, consectetur adipiscing elit. Sed cursus, ex vitae sagittis tempor, risus ipsum ultrices elit, vel eleifend magna ex in tellus. Proin congue mattis sem, eget gravida tellus iaculis nec. Sed consectetur varius quam convallis tincidunt. Aliquam congue augue vitae lacus dignissim, efficitur condimentum magna lacinia.

Etiam a convallis quam, in feugiat purus. Mauris porta magna ipsum, ac pharetra nisl tristique ac. Nullam consectetur finibus tortor, eget ultrices libero auctor a. Cras facilisis sapien metus, at accumsan ex auctor a. Pellentesque varius non dolor vitae rhoncus. Cras eget ante vitae magna aliquam condimentum. Vestibulum ante ipsum primis in faucibus orci luctus et ultrices posuere cubilia Curae; Donec molestie est consequat maximus dictum.\cite{exampleref1}

\section{Nunc Nec Ultrices Justo}

Nunc nec ultrices justo, nec rhoncus urna. Proin sagittis tempor purus, nec pulvinar leo varius eget. Etiam a nisi nec neque dignissim molestie. Ut tincidunt vel diam sed imperdiet. Praesent a auctor turpis, in semper ipsum. Sed eu erat interdum, sagittis lectus nec, fringilla felis. Nunc lacus lacus, scelerisque eu mollis ac, ultrices a leo. Aliquam mattis felis enim, in cursus augue dignissim eget. Aenean fermentum mi id facilisis pulvinar.\cite{exampleref1}

Aliquam aliquam, lorem elementum mollis ornare, augue felis tristique est, a blandit velit dui a orci. Sed id nisl dictum, tincidunt dui quis, gravida felis. Curabitur vitae erat at lacus volutpat iaculis ut ut enim. Nullam vel venenatis metus. Sed rutrum est at lectus elementum viverra. Nulla sed enim risus.

\chapter{Nulla Feugiat}

\begin{table}[h]
  \centering
  \begin{tabular}{l c r}
    1 & 2 & 3 \\
    4 & 5 & 6 \\
    7 & 8 & 9 \\
  \end{tabular}
  \caption{Lorem ipsum dolor sit amet, consectetur adipiscing elit. Nulla feugiat ante quis consectetur pellentesque. In tincidunt orci in justo tempor, non tempor metus congue.\cite{exampleref1}}
\end{table}

\subfile{Appendix}

\bibliographystyle{plain}
\renewcommand{\bibname}{References}
\bibliography{thesis}

\end{document}
